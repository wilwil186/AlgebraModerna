\documentclass[12pt]{article}
\usepackage[utf8]{inputenc}
\usepackage{amsmath, amssymb, amsthm}
\usepackage{enumitem}
\usepackage{geometry}
\geometry{margin=2cm}
\usepackage{hyperref} % Opcional: para hipervínculos en el PDF

% Entornos theorem, proposition, etc.
\newtheorem{theorem}{Teorema}[section]
\newtheorem{proposition}[theorem]{Proposición}
\newtheorem{lemma}[theorem]{Lema}
\newtheorem{corollary}[theorem]{Corolario}
\theoremstyle{definition}
\newtheorem{definition}[theorem]{Definición}
\theoremstyle{remark}
\newtheorem{remark}[theorem]{Observación}
\newtheorem{example}[theorem]{Ejemplo}

% Comandos para atajos
\newcommand{\Z}{\mathbb{Z}}
\newcommand{\Q}{\mathbb{Q}}
\newcommand{\R}{\mathbb{R}}
\newcommand{\C}{\mathbb{C}}

\title{Taller \# 2 de Anillos y Campos}
\author{
    Juliancito Vera (Código: \textbf{(xxxxxxxxxxxx)}), \\
    Nico Nicol (Código: \textbf{(xxxxxxxxxxxxx)}), \\
    y Wilson Jerez (Código: 201181167034)
}
\date{
    Universidad Distrital Francisco José de Caldas \\
    Facultad de Ciencias Matemáticas y Naturales \\
    Programa Académico de Matemáticas
}

\begin{document}

\maketitle

\section*{Ejercicios}

\begin{enumerate}
    \item Sea $f(x) = x^6 + 3x^5 + 4x^2 - 3x + 2$ y $g(x) = x^2 + 2x - 3$ en $\mathbb{Z}_7[x]$. Encuéntrese $q(x)$ y $r(x)$ en $\mathbb{Z}_7[x]$ tal que $f(x) = g(x)q(x) + r(x)$ con $\deg(r(x)) < 2$.
    
    \item El polinomio $x^4 + 4$ puede factorizarse en factores lineales en $\mathbb{Z}_5[x]$. Encuéntrese esta factorización.
    
    \item ¿Es $x^3 + 2x + 3$ un polinomio irreducible de $\mathbb{Z}_5[x]$? ¿Por qué? Exprésese como producto de polinomios irreducibles de $\mathbb{Z}_5[x]$.
    
    \item Pruebe que si $F$ es un campo, todo ideal primo propio de $F[x]$ es maximal.
    
    \item Si $D$ es un dominio de ideales principales (DIP), entonces $D[x]$ es un DIP.
    
    \item Indique cuáles de las funciones dadas $\nu$ son evaluaciones euclidianas para los dominios enteros dados.
    \begin{enumerate}
        \item La función $\nu$ para $\mathbb{Z}$ dada por $\nu(n) = n^2$ para $n \in \mathbb{Z}$ distinto de cero.
        \item La función $\nu$ para $\mathbb{Q}$ dada por $\nu(a) = a^2$ para $a \in \mathbb{Q}$ distinto de cero.
    \end{enumerate}
    
    \item Encuéntrese un mcd de los polinomios
    \begin{align*}
        &x^{10} - 3x^9 + 3x^8 - 11x^7 + 11x^6 - 11x^5 + 19x^4 - 13x^3 + 8x^2 - 9x+ 3, \\
        &x^6 - 3x^5 + 4x^4 - 9x^3 + 5x^2 - 5x+ 2
    \end{align*}
    en $\mathbb{Q}[x]$.
    
    \item Muéstrese que $\{a+ xf(x) \mid a \in \mathbb{Z}, f(x) \in \mathbb{Z}[x]\}$ es un ideal en $\mathbb{Z}[x]$.
    
    \item Sea $D$ un dominio euclidiano y sea $\nu$ una evaluación euclidiana en $D$. Muéstrese que si $a$ y $b$ son asociados en $D$, entonces $\nu(a) = \nu(b)$.
    
    \item Sea $D$ un DFU. Un elemento $c$ en $D$ es un mínimo común múltiplo de dos elementos $a$ y $b$ en $D$ si $a \mid c$ y $b \mid c$ y $c$ divide a todo elemento de $D$ que sea divisible entre $a$ y $b$. Muéstrese para cualesquiera dos elementos no nulos de $D$, un dominio euclidiano, tienen un mínimo común múltiplo en $D$.
    
    \item Considerando $\mathbb{Z}[\sqrt{-5}]$ como subanillo de los Complejos, defina para $z \in \mathbb{Z}[\sqrt{-5}]$ la función $N(z) = z\bar{z}$ y use esta para mostrar que $6$ no se factoriza de manera única (sin considerar asociados) en irreducibles en $\mathbb{Z}[\sqrt{-5}]$. Exhíbanses dos factorizaciones diferentes.
    
    \item Use el algoritmo euclideano en $\mathbb{Z}[i]$ para encontrar el máximo común divisor de $8 + 6i$ y $5 - 15i$.

    \item Sea $\langle \alpha \rangle$ un ideal principal distinto de cero en $\mathbb{Z}[i]$.
    
    \begin{enumerate}
        \item[a)] Muéstrese que $\mathbb{Z}[i]/\langle \alpha \rangle$ es un anillo finito. \textbf{[Sugerencia: úsese el algoritmo de división.]}
        
        \item[b)] Muéstrese que si $\pi$ es un irreducible de $\mathbb{Z}[i]$, entonces $\mathbb{Z}[i]/\langle \pi \rangle$ es un campo.
        
        \item[c)] Con respecto a b), encuéntrese el orden $n$ y característica de cada uno de los siguientes campos:
        
        \begin{enumerate}
            \item[1)] $\mathbb{Z}[i]/\langle 3 \rangle$
            \item[2)] $\mathbb{Z}[i]/\langle 1 + i \rangle$
            \item[3)] $\mathbb{Z}[i]/\langle 2 + i \rangle$
        \end{enumerate}
    \end{enumerate}

    \item Sea $n \in \mathbb{Z}^+$ libre de cuadrado, esto es, no es divisible entre el cuadrado de ningún entero primo. Sea $\mathbb{Z}[\sqrt{-n}] = \{ a + b\sqrt{-n} \mid a, b \in \mathbb{Z} \}$.
    
    \begin{enumerate}
        \item[a)] Defínase la norma $N$ definida por $N(a + b\sqrt{-n}) = a^2 + nb^2$ para $a + b\sqrt{-n}$ en su norma multiplicativa en $\mathbb{Z}[\sqrt{-n}]$.
        
        \item[b)] Muéstrese que $N(\alpha) = 1$ para $\alpha \in \mathbb{Z}[\sqrt{-n}]$ si y solo si $\alpha$ es una unidad en $\mathbb{Z}[\sqrt{-n}]$.
        
        \item[c)] Muéstrese que todo $\alpha \in \mathbb{Z}[\sqrt{-n}]$ distinto de cero que no sea unidad, tiene factorización en irreducibles en $\mathbb{Z}[\sqrt{-n}]$. \textbf{[Sugerencia: úsese b].}
    \end{enumerate}
\end{enumerate}
\section*{Ejercicios de la clase}
\begin{enumerate}
    \item Sea $D$ un dominio entero y $F$ su campo de fracciones. Entonces, para cualquier polinomio $f(X) \in F[X]$, existe un polinomio $f_0(X) \in D[X]$ y un elemento $a \in D$ tal que:
    \[
    f(X) = \frac{f_0(X)}{a}.
    \]
    \begin{itemize}

        \item Dado que $D$ es un dominio entero, su campo de fracciones $F$ consiste en todas las fracciones de la forma $\frac{a}{b}$, donde $a, b \in D$ y $b \neq 0$. Consideremos el anillo de polinomios $F[X]$, cuyos elementos son expresiones de la forma:
        \[
        f(X) = \sum_{i=0}^{n} c_i X^i, \quad \text{con } c_i \in F.
        \]
        Queremos demostrar que cualquier polinomio en $F[X]$ puede escribirse como $f(X) = \frac{f_0(X)}{a}$, donde $f_0(X) \in D[X]$ y $a \in D$.

        \item Construcción de $f_0(X)$: Dado un polinomio $f(X) \in F[X]$, podemos escribir cada coeficiente $c_i$ en términos de elementos de $D$:
        \[
        c_i = \frac{a_i}{b_i}, \quad \text{con } a_i, b_i \in D, \quad b_i \neq 0.
        \]
        Sea $a$ el \textbf{mínimo común múltiplo} de los denominadores $b_0, b_1, \dots, b_n$, es decir,
        \[
        a = \operatorname{mcm}(b_0, b_1, \dots, b_n) \in D.
        \]
        Por la propiedad del mínimo común múltiplo, sabemos que $a$ es un múltiplo de cada $b_i$, lo que significa que existe $k_i \in D$ tal que:
        \[
        a = k_i b_i.
        \]
        Multiplicamos ambos lados por $a_i$, obteniendo:
        \[
        a a_i = k_i b_i a_i.
        \]
        Ahora, dividiendo por $b_i$ (que es distinto de cero en $D$):
        \[
        \frac{a a_i}{b_i} = k_i a_i.
        \]
        Dado que $k_i, a_i \in D$ y $D$ es un anillo, el producto $k_i a_i$ también pertenece a $D$. Definiendo $d_i = k_i a_i$, obtenemos:
        \[
        d_i = \frac{a a_i}{b_i} \in D.
        \]
        Definimos entonces el polinomio $f_0(X)$ en $D[X]$ como:
        \[
        f_0(X) = \sum_{i=0}^{n} d_i X^i.
        \]
        Por construcción, tenemos:
        \[
        f(X) = \sum_{i=0}^{n} c_i X^i = \sum_{i=0}^{n} \frac{a_i}{b_i} X^i = \sum_{i=0}^{n} \frac{d_i}{a} X^i = \frac{1}{a} \sum_{i=0}^{n} d_i X^i = \frac{f_0(X)}{a}.
        \]

        \item Conclusión: Hemos demostrado que cualquier polinomio en $F[X]$ puede escribirse como $f(X) = \frac{f_0(X)}{a}$ con $f_0(X) \in D[X]$ y $a \in D$. Esto implica que $D[X]$ es un subanillo de $F[X]$, ya que cada polinomio en $F[X]$ se obtiene como un polinomio en $D[X]$ dividido por un elemento de $D$.
        \[
        \square
        \]
    \end{itemize}

    \item Muestre que el polinomio \( p(x) = x^2 + x + 3 \) es irreducible en \( \mathbb{Q}[x] \).
    \item Determine los elementos de \( \mathbb{Q}[x] / \langle p(x) \rangle \).
    \item Encuentre el inverso multiplicativo para \( a + b t \) en \( \mathbb{Q}[x] / \langle p(x) \rangle \) con \( a + b t \neq 0 \).

\end{enumerate}
\end{document}