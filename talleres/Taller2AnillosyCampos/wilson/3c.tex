Determine los elementos de \( \mathbb{Q}[x] / \langle p(x) \rangle \).

\begin{itemize}
    \item Los elementos de $\mathbb{Q}[x] / \langle p(x) \rangle$ están dados por las clases de equivalencia de los polinomios en $\mathbb{Q}[x]$ módulo $p(x)$. Formalmente, podemos describir estos elementos como sigue:
    
    El conjunto de clases de equivalencia es
    \[
    \mathbb{Q}[x] / \langle p(x) \rangle = \{ f(x) + \langle p(x) \rangle \mid f(x) \in \mathbb{Q}[x] \}.
    \]
    Es decir, dos polinomios $f(x)$ y $g(x)$ representan el mismo elemento si su diferencia es un múltiplo de $p(x)$, es decir, si $f(x) \equiv g(x) \mod p(x)$.
    
    Como $p(x)$ tiene grado $n$, cada clase de equivalencia tiene un representante único de la forma:
    \[
     a_0 + a_1 x + \dots + a_{n-1} x^{n-1},
    \]
    donde $a_i \in \mathbb{Q}$. Esto se debe a que cualquier polinomio $f(x)$ en $\mathbb{Q}[x]$ puede reducirse módulo $p(x)$ mediante la división euclidiana, dejando un residuo de grado menor que $n$.
    
    \begin{itemize}
        \item Si $p(x)$ es irreducible sobre $\mathbb{Q}$, entonces $\mathbb{Q}[x] / \langle p(x) \rangle$ es un \textbf{cuerpo} y se puede interpretar como una extensión de $\mathbb{Q}$ de grado $n$.
        \item Si $p(x)$ es reducible, el cociente es un \textbf{anillo con divisores de cero}, no necesariamente un cuerpo.
    \end{itemize}
    
    Para $p(x) = x^2 + 1$, los elementos del cociente son de la forma:
    \[
     a + bx, \quad a, b \in \mathbb{Q}.
    \]
    En este caso, $\mathbb{Q}[x] / \langle x^2 + 1 \rangle$ es isomorfo a $\mathbb{Q}(i)$, donde $i^2 = -1$, representando la extensión $\mathbb{Q}$ con la unidad imaginaria.
\end{itemize}