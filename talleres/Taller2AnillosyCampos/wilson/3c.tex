Determine los elementos de \( \mathbb{Q}[x] / \langle p(x) \rangle \).

Sea \(p(x)\) un polinomio de grado \(n\) en \(\mathbb{Q}[x]\). El anillo cociente 
\[
\mathbb{Q}[x]/\langle p(x)\rangle
\]
está formado por las clases de equivalencia de polinomios bajo la relación
\[
f(x) \sim g(x) \quad \Longleftrightarrow \quad f(x) - g(x) \in \langle p(x)\rangle,
\]
es decir, \(f(x) - g(x)\) es un múltiplo de \(p(x)\). A continuación, se describen sus elementos y estructura:

\begin{enumerate}
    \item \textbf{Elementos.} 
    Cada elemento (clase de equivalencia) puede representarse por un polinomio de grado menor que \(n\). 
    Esto se debe a que, dado cualquier polinomio \(f(x)\), al dividirlo por \(p(x)\) se obtiene
    \[
    f(x) = q(x)\,p(x) + r(x),
    \]
    donde \(\deg(r(x)) < n\). 
    El residuo \(r(x)\) es el representante único (de grado menor que \(\deg p(x)\)) de la clase de \(f(x)\). 
    Por tanto, cada clase de equivalencia puede escribirse como
    \[
    a_0 + a_1x + \dots + a_{n-1}x^{n-1}, \quad a_i\in \mathbb{Q}.
    \]

    \item \textbf{Operaciones.} 
    La suma y multiplicación se definen como en \(\mathbb{Q}[x]\), pero considerando que, 
    tras operar, se reduce el resultado módulo \(p(x)\). 
    En otras palabras, si \([f(x)]\) denota la clase de equivalencia de \(f(x)\), entonces:
    \[
    [f(x)] + [g(x)] \;=\; [\,f(x) + g(x)\,],
    \]
    \[
    [f(x)] \cdot [g(x)] \;=\; [\,f(x)\,g(x)\,].
    \]
    Al final, el resultado se reemplaza por su residuo de grado menor que \(n\).

    \item \textbf{Interpretación como espacio vectorial.} 
    Visto como espacio vectorial sobre \(\mathbb{Q}\), el anillo \(\mathbb{Q}[x]/\langle p(x)\rangle\) 
    tiene dimensión \(n\). Sus elementos pueden entenderse como “polinomios truncados” hasta grado \(n-1\), 
    con las operaciones inducidas por la reducción módulo \(p(x)\).

    \item \textbf{Caso especial: \(p(x)\) irreducible.} 
    Si \(p(x)\) es irreducible sobre \(\mathbb{Q}\), el ideal \(\langle p(x)\rangle\) es maximal, 
    por lo que el cociente \(\mathbb{Q}[x]/\langle p(x)\rangle\) es un \textbf{cuerpo}. 
    Esto se interpreta como una extensión de \(\mathbb{Q}\) de grado \(n\), frecuentemente denotada por 
    \(\mathbb{Q}(\alpha)\), donde \(\alpha\) es una raíz de \(p(x)\). 
    En este caso, todo elemento no nulo tiene inverso, y el cociente resulta muy útil en la construcción 
    de extensiones algebraicas de \(\mathbb{Q}\).
\end{enumerate}

\noindent
\textbf{Resumen.} Los elementos de \(\mathbb{Q}[x]/\langle p(x)\rangle\) son las clases de equivalencia 
de polinomios módulo \(p(x)\). Cada clase se representa de forma única por un polinomio de grado 
menor que \(n = \deg(p(x))\). La suma y el producto se definen módulo \(p(x)\). 
Si \(p(x)\) es irreducible, este anillo cociente se convierte en un cuerpo 
que cumple un rol central en la teoría de campos y en el estudio de extensiones algebraicas.

