Sea $n \in \mathbb{Z}^+$ libre de cuadrado, esto es, no es divisible por el cuadrado de ningún primo. Sea $\mathbb{Z}[\sqrt{-n}] = \{\,a + b\sqrt{-n}\mid a,b \in \mathbb{Z}\}$.
    
    \begin{enumerate}
        \item[a)] Defínase la norma $N$ dada por $N(a + b\sqrt{-n}) = a^2 + nb^2$, identificándola como una norma multiplicativa en $\mathbb{Z}[\sqrt{-n}]$.
        
        \item[b)] Muéstrese que $N(\alpha) = 1$ para $\alpha \in \mathbb{Z}[\sqrt{-n}]$ si y solo si $\alpha$ es una unidad en $\mathbb{Z}[\sqrt{-n}]$.
        
        \item[c)] Muéstrese que todo $\alpha \in \mathbb{Z}[\sqrt{-n}]$ que sea distinto de cero y no sea unidad tiene factorización en irreducibles en $\mathbb{Z}[\sqrt{-n}]$. \textbf{[Sugerencia: úsese (b).]}
    \end{enumerate}

\textbf{Solución} \\

\noindent
\textbf{(a) Definición de la norma y multiplicatividad}

Sea $\alpha = a + b\sqrt{-n}$ en $\mathbb{Z}[\sqrt{-n}]$. Definimos la norma
\[
N(\alpha) \;=\; a^2 + n\,b^2.
\]
Queremos ver que, dadas $\alpha = a + b\sqrt{-n}$ y $\beta = c + d\sqrt{-n}$, se cumple
\[
N(\alpha \beta) \;=\; N(\alpha)\, N(\beta).
\]
En efecto, si multiplicamos
\[
\alpha \beta = (a + b\sqrt{-n})(c + d\sqrt{-n}) = (ac - bdn) + (ad + bc)\sqrt{-n},
\]
entonces, al calcular
\[
N(\alpha \beta) = (ac - bdn)^2 \;+\; n\,(ad + bc)^2,
\]
y tras expandir con cuidado, podemos comprobar que
\[
(a^2 + nb^2)\,(c^2 + nd^2) 
\;=\; (ac - bdn)^2 \;+\; n\,(ad + bc)^2.
\]
Así, $N(\alpha\beta) = N(\alpha)\,N(\beta)$, confirmando que $N$ es un morfismo multiplicativo.

---

\noindent
\textbf{(b) Caracterización de las unidades mediante la norma}

Queremos mostrar que $N(\alpha) = 1$ si y sólo si $\alpha$ es una unidad en $\mathbb{Z}[\sqrt{-n}]$. 

\begin{itemize}
    \item[\(\Longrightarrow\)] Si $N(\alpha) = 1$, consideramos la inversa de $\alpha = a + b\sqrt{-n}$ en el campo de fracciones $\mathbb{Q}(\sqrt{-n})$. Se sabe que
    \[
    \alpha^{-1} \;=\; \frac{a - b\sqrt{-n}}{a^2 + n\,b^2}.
    \]
    Dado que $a^2 + n\,b^2 = 1$, la inversa se simplifica a $a - b\sqrt{-n}$, que está de nuevo en $\mathbb{Z}[\sqrt{-n}]$. Esto prueba directamente que $\alpha$ es invertible (es decir, es una unidad) en el anillo.
    
    \item[\(\Longleftarrow\)] Si $\alpha$ es unidad, existe alguna $\beta \in \mathbb{Z}[\sqrt{-n}]$ tal que $\alpha\beta = 1$. Aplicando la norma y usando su multiplicatividad,
    \[
    N(\alpha \beta) = N(\alpha)\,N(\beta) = N(1) = 1.
    \]
    Dado que $N(\alpha)$ y $N(\beta)$ son números enteros positivos (excepto si fueran cero, en cuyo caso no tendríamos una unidad), la única forma de que su producto sea 1 es que ambos valgan 1. Así, $N(\alpha)=1$.
\end{itemize}

En resumen, las unidades son exactamente aquellos elementos con norma igual a 1.

---

\noindent
\textbf{(c) Factorización de elementos no nulos ni unidades en irreducibles}

Para demostrar la factorización en irreducibles en $\mathbb{Z}[\sqrt{-n}]$, usamos el \textbf{principio de buena ordenación} en la norma.

\begin{itemize}
    \item Si $\alpha$ no es una unidad, entonces $N(\alpha) > 1$. Si $\alpha$ no es irreducible, se puede escribir como $\alpha = \beta \gamma$ con $\beta, \gamma$ no unidades.
    \item Como la norma es multiplicativa, tenemos que $N(\alpha) = N(\beta) N(\gamma)$, y por ser enteros positivos, se tiene $N(\beta), N(\gamma) < N(\alpha)$.
    \item Procedemos por inducción en la norma. Si todo elemento de norma menor que $N(\alpha)$ tiene factorización en irreducibles, entonces también lo tiene $\alpha$, pues sus factores $\beta$ y $\gamma$ se pueden descomponer en irreducibles.
    \item Aplicando el principio de buena ordenación, concluimos que todo elemento distinto de cero y no unidad en $\mathbb{Z}[\sqrt{-n}]$ se puede descomponer en irreducibles.
\end{itemize}

\[
\boxed{\text{Con esto, queda demostrada la factorización en irreducibles.}}
\]



