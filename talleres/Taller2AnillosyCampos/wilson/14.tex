Sea $n \in \mathbb{Z}^+$ libre de cuadrado, esto es, no es divisible por el cuadrado de ningún primo. Sea $\mathbb{Z}[\sqrt{-n}] = \{\,a + b\sqrt{-n}\mid a,b \in \mathbb{Z}\}$.
    
    \begin{enumerate}
        \item[a)] Defínase la norma $N$ dada por $N(a + b\sqrt{-n}) = a^2 + nb^2$, identificándola como una norma multiplicativa en $\mathbb{Z}[\sqrt{-n}]$.
        
        \item[b)] Muéstrese que $N(\alpha) = 1$ para $\alpha \in \mathbb{Z}[\sqrt{-n}]$ si y solo si $\alpha$ es una unidad en $\mathbb{Z}[\sqrt{-n}]$.
        
        \item[c)] Muéstrese que todo $\alpha \in \mathbb{Z}[\sqrt{-n}]$ que sea distinto de cero y no sea unidad tiene factorización en irreducibles en $\mathbb{Z}[\sqrt{-n}]$. \textbf{[Sugerencia: úsese (b).]}
    \end{enumerate}

\textbf{Solución} \\

\noindent
\textbf{(a) Definición de la norma y multiplicatividad}

Sea $\alpha = a + b\sqrt{-n}$ en $\mathbb{Z}[\sqrt{-n}]$. Definimos la norma
\[
N(\alpha) \;=\; a^2 + n\,b^2.
\]
Queremos ver que, dadas $\alpha = a + b\sqrt{-n}$ y $\beta = c + d\sqrt{-n}$, se cumple
\[
N(\alpha \beta) \;=\; N(\alpha)\, N(\beta).
\]
En efecto, si multiplicamos
\[
\alpha \beta = (a + b\sqrt{-n})(c + d\sqrt{-n}) = (ac - bdn) + (ad + bc)\sqrt{-n},
\]
entonces, al calcular
\[
N(\alpha \beta) = (ac - bdn)^2 \;+\; n\,(ad + bc)^2,
\]
y tras expandir con cuidado, podemos comprobar que
\[
(a^2 + nb^2)\,(c^2 + nd^2) 
\;=\; (ac - bdn)^2 \;+\; n\,(ad + bc)^2.
\]
Así, $N(\alpha\beta) = N(\alpha)\,N(\beta)$, confirmando que $N$ es un morfismo multiplicativo.

---

\noindent
\textbf{(b) Caracterización de las unidades mediante la norma}

Queremos mostrar que $N(\alpha) = 1$ si y sólo si $\alpha$ es una unidad en $\mathbb{Z}[\sqrt{-n}]$. 

\begin{itemize}
    \item[\(\Longrightarrow\)] Si $N(\alpha) = 1$, consideramos la inversa de $\alpha = a + b\sqrt{-n}$ en el campo de fracciones $\mathbb{Q}(\sqrt{-n})$. Se sabe que
    \[
    \alpha^{-1} \;=\; \frac{a - b\sqrt{-n}}{a^2 + n\,b^2}.
    \]
    Dado que $a^2 + n\,b^2 = 1$, la inversa se simplifica a $a - b\sqrt{-n}$, que está de nuevo en $\mathbb{Z}[\sqrt{-n}]$. Esto prueba directamente que $\alpha$ es invertible (es decir, es una unidad) en el anillo.
    
    \item[\(\Longleftarrow\)] Si $\alpha$ es unidad, existe alguna $\beta \in \mathbb{Z}[\sqrt{-n}]$ tal que $\alpha\beta = 1$. Aplicando la norma y usando su multiplicatividad,
    \[
    N(\alpha \beta) = N(\alpha)\,N(\beta) = N(1) = 1.
    \]
    Dado que $N(\alpha)$ y $N(\beta)$ son números enteros positivos (excepto si fueran cero, en cuyo caso no tendríamos una unidad), la única forma de que su producto sea 1 es que ambos valgan 1. Así, $N(\alpha)=1$.
\end{itemize}

En resumen, las unidades son exactamente aquellos elementos con norma igual a 1.

---

\noindent
\textbf{(c) Factorización de elementos no nulos ni unidades en irreducibles}

Ahora, tomemos $\alpha \in \mathbb{Z}[\sqrt{-n}]$ que sea distinto de cero y no unidad. Deseamos ver que $\alpha$ puede descomponerse en irreducibles. 

La idea fundamental se apoya en el hecho de que la norma $N(\alpha)$ es un número entero positivo. Si $\alpha$ no fuera irreducible, se expresaría como $\beta\gamma$ con $\beta,\gamma$ no unidades. En ese caso, tanto $N(\beta)$ como $N(\gamma)$ son estrictamente mayores que 1 pero menores que $N(\alpha)$. Esto reduce la norma y permite un proceso inductivo: se aplica la misma factorización a $\beta$ y $\gamma$, cuyos productos normales van descendiendo. Pero no podemos seguir disminuyendo indefinidamente una sucesión de números enteros positivos; por lo tanto, tarde o temprano este proceso de factorización debe terminar en factores que no se pueden seguir descomponiendo (más allá de unidades), es decir, en irreducibles. 

Con ello se concluye que cualquier elemento no nulo que no sea unidad admite al menos una factorización en irreducibles.

---

\noindent
\textbf{Resumen}

\begin{itemize}
    \item[(a)] Se define la norma $N(a+b\sqrt{-n}) = a^2 + n\,b^2$ y se verifica que es multiplicativa.
    \item[(b)] Un elemento $\alpha$ en $\mathbb{Z}[\sqrt{-n}]$ es unidad si y sólo si su norma es 1.
    \item[(c)] Cualquier elemento no nulo que no sea unidad puede descomponerse en irreducibles, gracias al principio de buena ordenación aplicado a las normas.
\end{itemize}

\[
\boxed{\text{Con esto, queda demostrada la factorización en irreducibles para los apartados (a), (b) y (c).}}
\]


