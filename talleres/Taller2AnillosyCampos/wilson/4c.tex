Encontrar el inverso multiplicativo de \( a + b t \) en \( \mathbb{Q}[x] / \langle p(x) \rangle \), donde \( p(x) \) es irreducible en \( \mathbb{Q}[x] \) y \( t \) es la clase de \( x \) en el cociente.

\textbf{Solución:} Como \( \mathbb{Q}[x] / \langle p(x) \rangle \) es un cuerpo, todo elemento no nulo tiene inverso. Se busca \( q(t) \) tal que:
\[
(a + b t) q(t) \equiv 1 \pmod{p(t)}.
\]
Dado que \( p(x) \) es irreducible y de grado \( n \), se cumple \( \gcd(a + b x, p(x)) = 1 \). Aplicando el algoritmo de Euclides extendido, existen \( q(x), k(x) \in \mathbb{Q}[x] \) tales que:
\[
(a + b x) q(x) + k(x) p(x) = 1.
\]
Reduciendo módulo \( p(x) \):
\[
(a + b x) q(x) \equiv 1 \pmod{p(x)}.
\]
Por lo tanto, \( q(t) \) es el inverso de \( a + b t \).

\textbf{Cálculo explícito para \( \deg(p) = 2 \):}  
Sea \( p(x) = x^2 + c x + d \), buscamos \( q(t) = u + v t \) tal que:
\[
(a + b t)(u + v t) \equiv 1 \pmod{p(t)}.
\]
Multiplicando:
\[
(a + b t)(u + v t) = a u + (a v + b u) t + b v t^2.
\]
Sustituyendo \( t^2 = -c t - d \):
\[
b v t^2 = -b v c t - b v d.
\]
\[
(a u - b v d) + (a v + b u - b v c) t \equiv 1.
\]
Sistema de ecuaciones:
\[
\begin{cases}
a u - b v d = 1, \\
a v + b u - b v c = 0.
\end{cases}
\]
Resolviendo:
\[
v = \frac{b}{a b c - a^2 - b^2 d}, \quad u = \frac{b c - a}{a b c - a^2 - b^2 d}.
\]
Inverso:
\[
q(t) = u + v t = \frac{b c - a}{a b c - a^2 - b^2 d} + \frac{b}{a b c - a^2 - b^2 d} t.
\]

\textbf{Caso general \( \deg(p) = n \):}  
El inverso \( q(t) = c_0 + c_1 t + \dots + c_{n-1} t^{n-1} \) se obtiene resolviendo el sistema de \( n \) ecuaciones que surge al imponer \( (a + b t) q(t) \equiv 1 \pmod{p(t)} \), expresando \( t^k \) en términos de \( 1, t, \dots, t^{n-1} \) usando \( p(t) = 0 \). Esto se resuelve mediante eliminación gaussiana o el algoritmo de Euclides extendido.

\textbf{Conclusión:}  
El inverso de \( a + b t \) en \( \mathbb{Q}[x]/\langle p(x) \rangle \) existe y es único, dado que el cociente es un cuerpo. Se obtiene aplicando el algoritmo de Euclides extendido o resolviendo un sistema de ecuaciones.

