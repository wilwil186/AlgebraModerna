Sea $f(x) = x^6 + 3x^5 + 4x^2 - 3x + 2$ y $g(x) = x^2 + 2x - 3$ en $\mathbb{Z}_7[x]$. 
Encuéntrese $q(x)$ y $r(x)$ en $\mathbb{Z}_7[x]$ tal que 
\[
f(x) = g(x)\,q(x) + r(x), 
\quad\text{con}\quad \deg(r(x)) < 2.
\]

\textbf{Solución:} Aplicamos la división de polinomios en $\mathbb{Z}_7[x]$, cuidando la aritmética módulo 7.

\begin{itemize}

    \item \textit{División inicial:} Dividimos el término de mayor grado de $f(x)$ entre el de mayor grado de $g(x)$:
    \[
    \frac{x^6}{x^2} \;=\; x^4.
    \]
    Multiplicamos $g(x)$ por $x^4$ y restamos:
    \[
    \begin{aligned}
    f(x) - x^4\,g(x)
    &=\; \bigl(x^6 + 3x^5 + 4x^2 - 3x + 2\bigr)
    \;-\; \bigl(x^6 + 2x^5 - 3x^4\bigr) \\
    &=\; (x^6 - x^6)
    + (3x^5 - 2x^5)
    + \bigl(0 - (-3x^4)\bigr)
    + 4x^2
    - 3x
    + 2 \\
    &=\; x^5 \;+\; 3x^4 \;+\; 4x^2 \;-\; 3x \;+\; 2.
    \end{aligned}
    \]
    Denotamos este nuevo polinomio como 
    \[
    r_1(x) = x^5 + 3x^4 + 4x^2 - 3x + 2.
    \]

    \item \textit{Segundo paso:} Dividimos el término de mayor grado de $r_1(x)$ entre $x^2$:
    \[
    \frac{x^5}{x^2} \;=\; x^3.
    \]
    Multiplicamos $g(x)$ por $x^3$ y restamos:
    \[
    \begin{aligned}
    r_1(x) - x^3\,g(x)
    &=\; \bigl(x^5 + 3x^4 + 4x^2 - 3x + 2\bigr)
    \;-\;\bigl(x^5 + 2x^4 - 3x^3\bigr) \\
    &=\; (x^5 - x^5)
    + (3x^4 - 2x^4)
    + \bigl(0 - (-3x^3)\bigr)
    + 4x^2
    - 3x
    + 2 \\
    &=\; x^4 + 3x^3 + 4x^2 - 3x + 2.
    \end{aligned}
    \]
    Sea 
    \[
    r_2(x) = x^4 + 3x^3 + 4x^2 - 3x + 2.
    \]

    \item \textit{Tercer paso:} Dividimos $x^4$ entre $x^2$:
    \[
    \frac{x^4}{x^2} \;=\; x^2.
    \]
    Multiplicamos $g(x)$ por $x^2$ y restamos de $r_2(x)$:
    \[
    \begin{aligned}
    r_2(x) - x^2\,g(x)
    &=\; \bigl(x^4 + 3x^3 + 4x^2 - 3x + 2\bigr)
    \;-\;\bigl(x^4 + 2x^3 - 3x^2\bigr) \\
    &=\; (x^4 - x^4)
    + (3x^3 - 2x^3)
    + (4x^2 - (-3x^2))
    - 3x
    + 2 \\
    &=\; x^3 + \bigl(4x^2 + 3x^2\bigr) - 3x + 2 \\
    &=\; x^3 + 7x^2 - 3x + 2 \\
    &\equiv\; x^3 - 3x + 2 
    \quad (\mathrm{mod}\;7),
    \end{aligned}
    \]
    porque $7x^2 \equiv 0$ en $\mathbb{Z}_7$. Denotamos 
    \[
    r_3(x) = x^3 - 3x + 2.
    \]

    \item \textit{Cuarto paso:} Dividimos $x^3$ entre $x^2$:
    \[
    \frac{x^3}{x^2} \;=\; x.
    \]
    Multiplicamos $g(x)$ por $x$ y restamos:
    \[
    \begin{aligned}
    r_3(x) - x\,g(x)
    &=\; \bigl(x^3 - 3x + 2\bigr)
    -\;\bigl(x^3 + 2x^2 - 3x\bigr) \\
    &=\; (x^3 - x^3) + \bigl(0\,x^2 - 2x^2\bigr) + \bigl((-3x)-(-3x)\bigr) + 2 \\
    &=\; -2x^2 + 2
    \;\equiv\; 5x^2 + 2
    \quad (\mathrm{mod}\,7).
    \end{aligned}
    \]
    Por tanto, ahora el resto es $5x^2 + 2$, que aún tiene grado 2, así que seguimos.

    \item \textit{Quinto paso:} Dividimos $5x^2$ entre $x^2$:
    \[
    \frac{5x^2}{x^2} \;=\; 5.
    \]
    Multiplicamos $g(x)$ por 5 (en $\mathbb{Z}_7$, $-2 \equiv 5$), y restamos:
    \[
    \begin{aligned}
    5 \cdot g(x)
    &=\; 5x^2 + 10x - 15
    \;\equiv\; 5x^2 + 3x + 6
    \quad (\mathrm{mod}\,7), \\
    \bigl(5x^2 + 2\bigr) \;-\; \bigl(5x^2 + 3x + 6\bigr)
    &=\; (5x^2-5x^2) + (0 - 3x) + (2 - 6) \\
    &=\; -3x - 4
    \;\equiv\; 4x + 3
    \quad (\mathrm{mod}\,7).
    \end{aligned}
    \]
    El resto final es, por tanto,
    \[
    r(x) = 4x + 3,
    \]
    y satisface $\deg\bigl(r(x)\bigr) < 2$.

\end{itemize}

Para hallar el cociente total $q(x)$, sumamos todos los términos usados en cada división:

\[
q(x) 
\;=\; x^4 \;+\; x^3 \;+\; x^2 \;+\; x \;+\; 5
\;\equiv\; x^4 + x^3 + x^2 + x - 2,
\quad (\text{en } \mathbb{Z}_7).
\]

\medskip
\noindent
\textbf{Conclusión:} Hemos obtenido
\[
\boxed{
q(x) = x^4 + x^3 + x^2 + x - 2
\quad\text{y}\quad
r(x) = 4x + 3.
}
\]
Verificando la igualdad $f(x) = g(x) \, q(x) + r(x)$ en $\mathbb{Z}_7[x]$, se confirma la corrección de esta división.




