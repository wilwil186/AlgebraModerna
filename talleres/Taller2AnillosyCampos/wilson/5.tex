 Si \( D \) es un dominio íntegro de ideales principales (DIP), entonces \( D[x] \) también lo es.

\textbf{Contraejemplo:} Consideremos \( D = \mathbb{Z} \), el cual es un DIP, ya que sus ideales son de la forma \( \langle n \rangle \) con \( n \in \mathbb{Z} \).

Ahora, consideremos el ideal
\[
I = \langle 2, x \rangle = \{ 2f(x) + xg(x) \mid f(x), g(x) \in \mathbb{Z}[x] \}.
\]

Si \( I \) fuera principal, existiría un polinomio \( h(x) \in \mathbb{Z}[x] \) tal que \( \langle h(x) \rangle = \langle 2, x \rangle \). Esto implicaría que \( h(x) \) divide a \( 2 \) y a \( x \). Los únicos divisores de \( 2 \) en \( \mathbb{Z}[x] \) son \( \pm 1, \pm 2 \), por lo que:

- Si \( h(x) = \pm 1 \), entonces \( \langle h(x) \rangle = \mathbb{Z}[x] \), lo cual es un absurdo, pues \( I \neq \mathbb{Z}[x] \) (ya que \( 1 \notin I \)).
- Si \( h(x) = \pm 2 \), entonces \( h(x) \) no divide a \( x \) (recordemos que x puede ser impar), lo que contradice el hecho de que \( h(x) \) debe generar \( x \).

Por lo tanto, se concluye que \( I = \langle 2, x \rangle \) no es principal, lo que demuestra que \( \mathbb{Z}[x] \) no es un DIP, a pesar de que \( \mathbb{Z} \) sí lo es.
