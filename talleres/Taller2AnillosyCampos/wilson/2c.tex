Muestre que el polinomio \( p(x) = x^2 + x + 3 \) es irreducible en \( \mathbb{Q}[x] \).

\textbf{Solución:}

Para demostrar la irreducibilidad del polinomio \( p(x) = x^2 + x + 3 \) en \( \mathbb{Q}[x] \), utilizamos el siguiente resultado:

\textbf{Teorema 23.10 (Fraleigh, 7ma edición)}\\
\textit{
Sea \( f(x) \in F[x] \) y supongamos que \( f(x) \) tiene grado 2 o 3. Entonces \( f(x) \) es reducible sobre \( F \) si y solo si tiene una raíz en \( F \).
}

\textbf{Demostración:}\\
Supongamos que \( f(x) \) es reducible sobre \( F \). Entonces puede escribirse como el producto de dos polinomios no constantes en \( F[x] \), es decir,
\[
f(x) = g(x)\,h(x),
\]
donde \(\deg g(x) < \deg f(x)\) y \(\deg h(x) < \deg f(x)\). 
Dado que \( f(x) \) tiene grado 2 o 3, uno de los factores (por ejemplo, \( g(x) \)) debe ser de grado 1. Por lo tanto, 
\[
g(x) = x - a, \quad \text{para algún } a \in F.
\]
Como \( g(a) = 0 \), concluimos que \( a \) es una raíz de \( f(x) \). 
De esta manera, si \( f(x) \) es reducible sobre \( F[x] \), necesariamente tiene una raíz en \( F \).

Recíprocamente, si existe \( a \in F \) tal que \( f(a) = 0 \), entonces \( x - a \) es un factor de \( f(x) \), lo que muestra que \( f(x) \) es reducible.

\textbf{Aplicación al ejercicio:}\\
Para comprobar que \( p(x) = x^2 + x + 3 \) es irreducible en \(\mathbb{Q}[x]\), basta con verificar que no tiene raíces racionales. 
Resolviendo la ecuación cuadrática asociada,
\[
x = \frac{-1 \pm \sqrt{1 - 12}}{2} 
   = \frac{-1 \pm \sqrt{-11}}{2},
\]
se observa que \(\sqrt{-11}\notin \mathbb{Q}\). 
Por lo tanto, \( p(x) \) no posee raíces en \(\mathbb{Q}\) y, de acuerdo con el Teorema 23.10, es irreducible sobre \(\mathbb{Q}\).
