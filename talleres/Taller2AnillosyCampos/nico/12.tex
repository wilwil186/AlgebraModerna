Use el algoritmo euclideano en $\mathbb{Z}[i]$ para encontrar el máximo común divisor de $8 + 6i$ y $5 - 15i$.\\
    \textbf{\textit{Soluci\'on:}}\\
    Sean $\alpha_1=5-15i$ y $\beta_1=8+6i$ en $\mathbb{Z}[i]$:\\
    $$\dfrac{5-15i}{8+6i}=\dfrac{5-15i}{8+6i}*\dfrac{8-6i}{8-6i}=\dfrac{(40-90)+(-30-120)i}{8^2+6^2}=\dfrac{-50-150i}{100}=-\dfrac{1}{2}-\dfrac{3}{2}i$$
    T\'omese $q_1,q_2\in \mathbb{Z}$ tal que: 
    $$\left|-\dfrac{1}{2}-q_1\right|\leq\dfrac{1}{2}\hspace{1cm}y\hspace{1cm}\left|-\dfrac{3}{2}-q_2\right|\leq\dfrac{1}{2}$$
    Donde $\theta=q_1+q_2i$. Entonces $q_1=0$ y $q_2=-1$, con lo cual $\theta=-i$. Por el algoritmo de euclides $\alpha_1=\beta_1\theta+p$ entonces
    $$p=\alpha_1-\theta\beta_{1}=(5-15i)-(-i)(8+6i)=-1-7i$$
    Entonces $5+15i=(8+6i)(-i)+(-1-7i)$. \\
    Siguiendo con el algoritmo euclideano, t\'omese: 
    $$\alpha_2=8+6i\hspace{2cm}y\hspace{2cm}\beta_2=-1-7i$$
    $$\dfrac{8+6i}{-1-7i}=\dfrac{8+6i}{-1-7i}*\dfrac{-1+7i}{1+7i}=\dfrac{-50+50i}{50}=-1+i$$
    Como $-1+i\in\mathbb{Z}[i]$, as\'i terminese el proceso. En conclusi\'on el maximo com\'un divisor entre $5+15i$ y $8+6i$ es $-1-7i$ \\
    Adem\'as multiplicando $-1-7i$ por las unidades de $\mathbb{Z}[i]$:
    $$(-1-7i)(-1)=1+7i$$
    $$(-1-7i)(-i)=-7+i$$
    $$(-1-7i)(i)=7-i$$
    Con lo cual $1+7i,-7+i,7-i$ también son mcd de $5-15i$ y $8-6i$