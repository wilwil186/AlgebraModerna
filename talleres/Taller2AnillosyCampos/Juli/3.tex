\noindent
Vamos a revisar detalladamente la factorización del polinomio \( f(x) = x^3 + 2x + 3 \) en \( \mathbb{Z}_5[x] \) y verificar si la factorización correcta es \( (x - 2)(x + 1)^2 \).

\smallskip

\textbf{Paso 1: Comprobación de raíces en \( \mathbb{Z}_5 \)}

Buscamos valores en \( \mathbb{Z}_5 = \{0,1,2,3,4\} \) que satisfagan \( f(x) = 0 \).

\[
f(x) = x^3 + 2x + 3
\]

Calculamos:

\[
\begin{aligned}
    f(0) &= 0^3 + 2(0) + 3 = 3 \neq 0, \\
    f(1) &= 1^3 + 2(1) + 3 = 1 + 2 + 3 = 6 \equiv 1 \pmod{5}, \\
    f(2) &= 2^3 + 2(2) + 3 = 8 + 4 + 3 = 15 \equiv 0 \pmod{5}, \quad \checkmark \\
    f(3) &= 3^3 + 2(3) + 3 = 27 + 6 + 3 = 36 \equiv 1 \pmod{5}, \\
    f(4) &= 4^3 + 2(4) + 3 = 64 + 8 + 3 = 75 \equiv 0 \pmod{5}, \quad \checkmark
\end{aligned}
\]

Encontramos que \( x = 2 \) y \( x = 4 \equiv -1 \pmod{5} \) son raíces.

\smallskip

\textbf{Paso 2: División de \( f(x) \) por \( (x - 2) \)}

Hacemos la división de \( f(x) \) entre \( (x - 2) \) en \( \mathbb{Z}_5 \).

El cociente es:

\[
x^2 + 2x + 1.
\]

\smallskip

\textbf{Paso 3: Factorización de \( x^2 + 2x + 1 \)}

\[
x^2 + 2x + 1 = (x + 1)(x + 1) = (x + 1)^2.
\]

\smallskip

\textbf{Conclusión}

La factorización completa de \( f(x) \) en \( \mathbb{Z}_5[x] \) es:

\[
(x - 2)(x + 1)^2.
\]

Esto muestra que \( f(x) \) \textbf{no es irreducible} en \( \mathbb{Z}_5[x] \) porque se descompone en factores de grado menor.
