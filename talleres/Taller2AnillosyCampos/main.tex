\documentclass[12pt]{article}
\usepackage[utf8]{inputenc}
\usepackage{amsmath, amssymb, amsthm}
\usepackage{enumitem}
\usepackage{geometry}
\geometry{margin=2cm}
\usepackage{hyperref} % Opcional: para hipervínculos en el PDF

% Entornos theorem, proposition, etc.
\newtheorem{theorem}{Teorema}[section]
\newtheorem{proposition}[theorem]{Proposición}
\newtheorem{lemma}[theorem]{Lema}
\newtheorem{corollary}[theorem]{Corolario}
\theoremstyle{definition}
\newtheorem{definition}[theorem]{Definición}
\theoremstyle{remark}
\newtheorem{remark}[theorem]{Observación}
\newtheorem{example}[theorem]{Ejemplo}

% Comandos para atajos
\newcommand{\Z}{\mathbb{Z}}
\newcommand{\Q}{\mathbb{Q}}
\newcommand{\R}{\mathbb{R}}
\newcommand{\C}{\mathbb{C}}

\title{Taller \# 2 de Anillos y Campos}
\author{
    Julián Vera (Código: (20212167064)), \\
    Nicole Vargas (Código: (20212167015)), \\
    y Wilson Jerez (Código: 201181167034)
}
\date{
    Universidad Distrital Francisco José de Caldas \\
    Facultad de Ciencias Matemáticas y Naturales \\
    Programa Académico de Matemáticas
}

\begin{document}

\maketitle

\section*{Ejercicios}

\begin{enumerate}
    \item \begin{itemize}
    \item Para determinar si el polinomio $ x^2 + 3 $ es irreducible en $ \mathbb{Z}_7 $, podemos usar el siguiente criterio:
    
    \textbf{Teorema (Fraleigh, 7\textsuperscript{a} edición, Teorema 23.10):}  
    Un polinomio cuadrático o cúbico $ f(x) $ sobre un campo finito $ \mathbb{F} $ es reducible si y solo si tiene una raíz en $ \mathbb{F} $.

    \textbf{Paso 1: Evaluar el polinomio en los elementos de $ \mathbb{Z}_7 $}

    Calculamos $ f(a) = a^2 + 3 $ para $ a \in \mathbb{Z}_7 $:

    \[
    \begin{aligned}
    f(0) &= 0^2 + 3 = 3 \not\equiv 0 \pmod{7} \\
    f(1) &= 1^2 + 3 = 4 \not\equiv 0 \pmod{7} \\
    f(2) &= 2^2 + 3 = 4 + 3 = 7 \equiv 0 \pmod{7} \\
    f(3) &= 3^2 + 3 = 9 + 3 = 12 \equiv 5 \pmod{7} \\
    f(4) &= 4^2 + 3 = 16 + 3 = 19 \equiv 5 \pmod{7} \\
    f(5) &= 5^2 + 3 = 25 + 3 = 28 \equiv 0 \pmod{7} \\
    f(6) &= 6^2 + 3 = 36 + 3 = 39 \equiv 4 \pmod{7} \\
    \end{aligned}
    \]

    \textbf{Paso 2: Concluir sobre la reducibilidad}

    Observamos que $ f(2) \equiv 0 \pmod{7} $ y $ f(5) \equiv 0 \pmod{7} $, lo que significa que $ x^2 + 3 $ tiene raíces en $ \mathbb{Z}_7 $, específicamente $ x = 2 $ y $ x = 5 $. Según el teorema citado, esto implica que $ x^2 + 3 $ es \textbf{reducible} en $ \mathbb{Z}_7 $.

    Por lo tanto, el polinomio $ x^2 + 3 $ \textbf{no es irreducible en} $ \mathbb{Z}_7 $.
\end{itemize}
    
    \item El polinomio $x^4 + 4$ puede factorizarse en factores lineales en $\mathbb{Z}_5[x]$. Encuéntrese esta factorización.
    
    \item ¿Es $x^3 + 2x + 3$ un polinomio irreducible de $\mathbb{Z}_5[x]$? ¿Por qué? Exprésese como producto de polinomios irreducibles de $\mathbb{Z}_5[x]$.
    
    \item Pruebe que si $F$ es un campo, todo ideal primo propio de $F[x]$ es maximal.
    
    \item  Si \( D \) es un dominio íntegro de ideales principales (DIP), entonces \( D[x] \) también lo es.

\textbf{Contraejemplo:} Consideremos \( D = \mathbb{Z} \), el cual es un DIP, ya que sus ideales son de la forma \( \langle n \rangle \) con \( n \in \mathbb{Z} \).

Ahora, consideremos el ideal
\[
I = \langle 2, x \rangle = \{ 2f(x) + xg(x) \mid f(x), g(x) \in \mathbb{Z}[x] \}.
\]

Si \( I \) fuera principal, existiría un polinomio \( h(x) \in \mathbb{Z}[x] \) tal que \( \langle h(x) \rangle = \langle 2, x \rangle \). Esto implicaría que \( h(x) \) divide a \( 2 \) y a \( x \). Los únicos divisores de \( 2 \) en \( \mathbb{Z}[x] \) son \( \pm 1, \pm 2 \), por lo que:

- Si \( h(x) = \pm 1 \), entonces \( \langle h(x) \rangle = \mathbb{Z}[x] \), lo cual es un absurdo, pues \( I \neq \mathbb{Z}[x] \) (ya que \( 1 \notin I \)).
- Si \( h(x) = \pm 2 \), entonces \( h(x) \) no divide a \( x \) (recordemos que x puede ser impar), lo que contradice el hecho de que \( h(x) \) debe generar \( x \).

Por lo tanto, se concluye que \( I = \langle 2, x \rangle \) no es principal, lo que demuestra que \( \mathbb{Z}[x] \) no es un DIP, a pesar de que \( \mathbb{Z} \) sí lo es.

    
    \item Indique cuáles de las funciones dadas $\nu$ son evaluaciones euclidianas para los dominios enteros dados.
    \begin{enumerate}
        \item La función $\nu$ para $\mathbb{Z}$ dada por $\nu(n) = n^2$ para $n \in \mathbb{Z}$ distinto de cero.
        \item La función $\nu$ para $\mathbb{Q}$ dada por $\nu(a) = a^2$ para $a \in \mathbb{Q}$ distinto de cero.
    \end{enumerate}
    
    \item Encuéntrese el mcd de los polinomios
\[
f(x) \;=\; x^{10} \;-\; 3x^9 \;+\; 3x^8 \;-\; 11x^7 \;+\; 11x^6 \;-\; 11x^5 
        \;+\; 19x^4 \;-\; 13x^3 \;+\; 8x^2 \;-\; 9x \;+\; 3,
\]
\[
g(x) \;=\; x^6 \;-\; 3x^5 \;+\; 4x^4 \;-\; 9x^3 \;+\; 5x^2 \;-\; 5x \;+\; 2
\]
en $\mathbb{Q}[x]$.

\textbf{Solución:} 

Aplicamos el \textbf{algoritmo de Euclides} siguiendo la sucesión típica de divisiones:
\[
\begin{aligned}
    f(x) &= q_1(x)\,g(x) + r_1(x), \\
    g(x) &= q_2(x)\,r_1(x) + r_2(x), \\
    r_1(x) &= q_3(x)\,r_2(x) + r_3(x), \\
          &\;\;\vdots \\
    r_{n-1}(x) &= q_n(x)\,r_n(x) + 0,
\end{aligned}
\]
donde el último residuo no nulo, \(r_n(x)\), es el \textbf{máximo común divisor}.

\medskip

En nuestro caso concreto, los pasos de división se especifican como sigue:

\[
\begin{aligned}
f(x) \;=\;& 
  (x^4 \;-\; 2x)\;\cdot g(x) 
  \;+\; \underbrace{\bigl(-2x^7 + 6x^6 - 6x^5 + 6x^4 - 13x^3 + 8x^2 - 9x + 3\bigr)}_{r_1(x)},
\\[4pt]
g(x) \;=\;& 
  (x^2 + 6x \;-\; 19)\;\cdot r_1(x) 
  \;+\; \underbrace{\bigl(19x^4 + 57x^3 + 38x^2 - 23x + 2\bigr)}_{r_2(x)},
\\[4pt]
r_1(x) \;=\;& 
  (\,x \;-\; 3\,)\;\cdot r_2(x) 
  \;+\; \underbrace{\bigl(x^3 + 2x - 1\bigr)}_{r_3(x)},
\\[4pt]
r_2(x) \;=\;& (19x+57 )\;\cdot r_3(x) \;+\; 0.
\end{aligned}
\]

Tras la última división, el proceso de Euclides concluye porque el residuo es cero y, por tanto, 
el \textit{último resto distinto de cero} es
\[
r_3(x) \;=\; x^3 \;+\; 2x \;-\; 1.
\]

En un anillo de polinomios sobre un campo $\mathbb{Q}[x]$, los divisores máximos comunes son únicos \emph{salvo} un factor constante no nulo. Así, concluimos:

\[
\boxed{\gcd\!\bigl(f(x),\,g(x)\bigr) \;=\; x^3 \;+\; 2x \;-\; 1.}
\]

\noindent
\textbf{Observación:} Cada coeficiente se maneja sobre \(\mathbb{Q}\), por lo que las operaciones de división de polinomios se realizan sin restricciones, y no necesitamos normalizar factores adicionales más allá de un posible factor multiplicativo no cero. Así queda verificado el resultado final. 

    
    \item Muéstrese que $\{a+ xf(x) \mid a \in \mathbb{Z}, f(x) \in \mathbb{Z}[x]\}$ es un ideal en $\mathbb{Z}[x]$.
    
    \item Sea $D$ un dominio euclidiano y sea $\nu$ una evaluación euclidiana en $D$. Muéstrese que si $a$ y $b$ son asociados en $D$, entonces $\nu(a) = \nu(b)$.
    
    \item Sea $D$ un DFU. Un elemento $c$ en $D$ es un mínimo común múltiplo de dos elementos $a$ y $b$ en $D$ si $a \mid c$ y $b \mid c$ y $c$ divide a todo elemento de $D$ que sea divisible entre $a$ y $b$. Muéstrese para cualesquiera dos elementos no nulos de $D$, un dominio euclidiano, tienen un mínimo común múltiplo en $D$.
    
    \item  \noindent
\textbf{Solución:}

\medskip

\noindent
\textbf{1. El anillo y la norma}

\noindent
Recordemos que
\[
\mathbb{Z}[\sqrt{-5}] \;=\; \{\, a + b\sqrt{-5} \;\mid\; a,b\in\mathbb{Z}\},
\]
y que para cada \(z = a + b\sqrt{-5}\) definimos la \emph{norma} como
\[
N(z) \;=\; z \, \overline{z} 
\;=\; (a + b\sqrt{-5})(a - b\sqrt{-5})
\;=\; a^2 + 5\,b^2.
\]
Esta norma resulta crucial porque \emph{es multiplicativa}, es decir, 
\[
N(z_1 z_2) \;=\; N(z_1)\,N(z_2),
\]
lo que nos ayudará a analizar la irreducibilidad de varios elementos.

\bigskip

\noindent
\textbf{2. Dos factorizaciones distintas de \(6\)}

\noindent
En \(\mathbb{Z}[\sqrt{-5}]\), el número entero \(6\) tiene las siguientes dos factorizaciones:
\[
6 \;=\; 2 \cdot 3
\quad\text{y}\quad
6 \;=\; (1 + \sqrt{-5})(1 - \sqrt{-5}).
\]
Vamos a ver que los factores que aparecen en una y otra expresión son \emph{irreducibles} y no se pueden relacionar por unidades (asociados). De este modo, comprobamos que la factorización de \(6\) en este anillo \emph{no} es única (hasta unidades).

\bigskip

\noindent
\textbf{3. Verificación de irreducibilidad de los factores}

\medskip

\noindent
\textbf{3.1. Irreducibilidad de \(2\)}

\begin{itemize}
    \item \emph{Norma de 2:} \(N(2) = 4\).
    \item Si \(2\) fuera reducible, existiría una factorización 
    \[
    2 \;=\; (a + b\sqrt{-5})(c + d\sqrt{-5}),
    \]
    con ninguno de los dos factores igual a \(\pm 1\) (los únicos posibles valores de las unidades en este anillo).  
    \item Tomando la norma, 
    \[
    4 = N(2) = N(a + b\sqrt{-5})\, N(c + d\sqrt{-5}).
    \]
    Eso implica que el par de normas debe multiplicarse para dar \(4\). En particular, podría pensarse en factorizar \(4\) como \(1 \times 4\), \(2 \times 2\) o \(4 \times 1\).
    \item \emph{Norma 2 imposible:} No hay solución en enteros para \(a^2 + 5 b^2 = 2\), pues revisando casos sencillos \((a,b)\) no aparece ninguna pareja que cumpla esa ecuación.  
    \item De modo que, si uno de los factores tuviera norma \(4\), el otro forzosamente tendría norma \(1\) (es decir, sería unidad). Esto demuestra que no podemos factorizarlos ambos como no unidades. Por lo tanto, \(2\) es irreducible.
\end{itemize}

\medskip

\noindent
\textbf{3.2. Irreducibilidad de \(3\)}

\begin{itemize}
    \item \emph{Norma de 3:} \(N(3) = 9\).
    \item Si \(3\) fuera reducible, al tomar la norma veríamos que la única forma de factorizar \(9\) con factores mayores que 1 es \(3 \times 3\). Sin embargo, no existe elemento en \(\mathbb{Z}[\sqrt{-5}]\) con norma \(3\), porque la ecuación \(a^2 + 5b^2 = 3\) tampoco tiene soluciones en enteros.
    \item Luego, si uno de los factores de la factorización hipotética de \(3\) no fuera unidad, su norma tendría que ser \(3\), lo cual no es posible. Así, no hay factorización no trivial. De ahí se concluye que \(\,3\) es irreducible.
\end{itemize}

\medskip

\noindent
\textbf{3.3. Irreducibilidad de \(1 + \sqrt{-5}\) y \(1 - \sqrt{-5}\)}

\begin{itemize}
    \item \emph{Normas:} 
    \[
    N(1 + \sqrt{-5}) = 1^2 + 5\cdot1^2 = 6,
    \quad
    N(1 - \sqrt{-5}) = 1^2 + 5\cdot1^2 = 6.
    \]
    \item Para factorizar, por ejemplo, \(1 + \sqrt{-5}\) en un producto no trivial \((x)(y)\), las normas de \(x\) e \(y\) tendrían que multiplicarse para dar \(6\). Por tanto, una de las normas debería ser \(2\) o \(3\) (porque \(6 = 2 \times 3\)), o bien \(1\) y \(6\). Pero ya hemos visto que no puede haber un factor con norma \(2\) ni con norma \(3\), y si uno de los factores tuviera norma \(1\), sería una unidad.  
    \item Por lo tanto, \(1 + \sqrt{-5}\) no admite factorizaciones no triviales (análogamente para \(1 - \sqrt{-5}\)). Esto prueba su irreducibilidad.
\end{itemize}

\bigskip

\noindent
\textbf{4. Diferencia esencial entre las dos factorizaciones de \(6\)}

\noindent
Hemos verificado que \(2\), \(3\), \(1 + \sqrt{-5}\) y \(1 - \sqrt{-5}\) son irreducibles. Ahora, para ver que las dos factorizaciones
\[
6 = 2 \cdot 3
\quad\text{y}\quad
6 = (1 + \sqrt{-5})(1 - \sqrt{-5})
\]
no son “la misma” (ni difieren sólo por una unidad), basta notar que no podemos convertir, por ejemplo, \(2\) en \(1 + \sqrt{-5}\) multiplicándola por \(\pm1\). Si existiera \(u \in \{\pm 1\}\) tal que 
\[
2 = u\, (1 + \sqrt{-5}),
\]
se obtendría una contradicción al comparar partes reales e imaginarias.  
Por tanto, estas factorizaciones no se relacionan por asociados, lo que confirma que \(\mathbb{Z}[\sqrt{-5}]\) no tiene factorización única.

\bigskip

\noindent
\textbf{5. Conclusión}

\noindent
Así, el elemento \(6\) en \(\mathbb{Z}[\sqrt{-5}]\) admite dos descomposiciones distintas en irreducibles:  
\[
6 \;=\; 2 \cdot 3 
\quad\quad\text{y}\quad\quad 
6 \;=\; (1 + \sqrt{-5})(1 - \sqrt{-5}),
\]
sin que los factores aparecidos en una factorización sean meramente asociados a los de la otra. Con esto finalizamos la demostración de que \(\mathbb{Z}[\sqrt{-5}]\) \emph{no} es un dominio de factorización única.

    
    \item Use el algoritmo euclideano en $\mathbb{Z}[i]$ para encontrar el máximo común divisor de $8 + 6i$ y $5 - 15i$.

    \item Sea $\langle \alpha \rangle$ un ideal principal distinto de cero en $\mathbb{Z}[i]$.
    
    \begin{enumerate}
        \item[a)] Muéstrese que $\mathbb{Z}[i]/\langle \alpha \rangle$ es un anillo finito. \textbf{[Sugerencia: úsese el algoritmo de división.]}
        
        \item[b)] Muéstrese que si $\pi$ es un irreducible de $\mathbb{Z}[i]$, entonces $\mathbb{Z}[i]/\langle \pi \rangle$ es un campo.
        
        \item[c)] Con respecto a b), encuéntrese el orden $n$ y característica de cada uno de los siguientes campos:
        
        \begin{enumerate}
            \item[1)] $\mathbb{Z}[i]/\langle 3 \rangle$
            \item[2)] $\mathbb{Z}[i]/\langle 1 + i \rangle$
            \item[3)] $\mathbb{Z}[i]/\langle 2 + i \rangle$
        \end{enumerate}
    \end{enumerate}

    \item Sea $n \in \mathbb{Z}^+$ libre de cuadrado, esto es, no es divisible por el cuadrado de ningún primo. Sea $\mathbb{Z}[\sqrt{-n}] = \{\,a + b\sqrt{-n}\mid a,b \in \mathbb{Z}\}$.
    
    \begin{enumerate}
        \item[a)] Defínase la norma $N$ dada por $N(a + b\sqrt{-n}) = a^2 + nb^2$, identificándola como una norma multiplicativa en $\mathbb{Z}[\sqrt{-n}]$.
        
        \item[b)] Muéstrese que $N(\alpha) = 1$ para $\alpha \in \mathbb{Z}[\sqrt{-n}]$ si y solo si $\alpha$ es una unidad en $\mathbb{Z}[\sqrt{-n}]$.
        
        \item[c)] Muéstrese que todo $\alpha \in \mathbb{Z}[\sqrt{-n}]$ que sea distinto de cero y no sea unidad tiene factorización en irreducibles en $\mathbb{Z}[\sqrt{-n}]$. \textbf{[Sugerencia: úsese (b).]}
    \end{enumerate}

\textbf{Solución} \\

\noindent
\textbf{(a) Definición de la norma y multiplicatividad}

Sea $\alpha = a + b\sqrt{-n}$ en $\mathbb{Z}[\sqrt{-n}]$. Definimos la norma
\[
N(\alpha) \;=\; a^2 + n\,b^2.
\]
Queremos ver que, dadas $\alpha = a + b\sqrt{-n}$ y $\beta = c + d\sqrt{-n}$, se cumple
\[
N(\alpha \beta) \;=\; N(\alpha)\, N(\beta).
\]
En efecto, si multiplicamos
\[
\alpha \beta = (a + b\sqrt{-n})(c + d\sqrt{-n}) = (ac - bdn) + (ad + bc)\sqrt{-n},
\]
entonces, al calcular
\[
N(\alpha \beta) = (ac - bdn)^2 \;+\; n\,(ad + bc)^2,
\]
y tras expandir con cuidado, podemos comprobar que
\[
(a^2 + nb^2)\,(c^2 + nd^2) 
\;=\; (ac - bdn)^2 \;+\; n\,(ad + bc)^2.
\]
Así, $N(\alpha\beta) = N(\alpha)\,N(\beta)$, confirmando que $N$ es un morfismo multiplicativo.

---

\noindent
\textbf{(b) Caracterización de las unidades mediante la norma}

Queremos mostrar que $N(\alpha) = 1$ si y sólo si $\alpha$ es una unidad en $\mathbb{Z}[\sqrt{-n}]$. 

\begin{itemize}
    \item[\(\Longrightarrow\)] Si $N(\alpha) = 1$, consideramos la inversa de $\alpha = a + b\sqrt{-n}$ en el campo de fracciones $\mathbb{Q}(\sqrt{-n})$. Se sabe que
    \[
    \alpha^{-1} \;=\; \frac{a - b\sqrt{-n}}{a^2 + n\,b^2}.
    \]
    Dado que $a^2 + n\,b^2 = 1$, la inversa se simplifica a $a - b\sqrt{-n}$, que está de nuevo en $\mathbb{Z}[\sqrt{-n}]$. Esto prueba directamente que $\alpha$ es invertible (es decir, es una unidad) en el anillo.
    
    \item[\(\Longleftarrow\)] Si $\alpha$ es unidad, existe alguna $\beta \in \mathbb{Z}[\sqrt{-n}]$ tal que $\alpha\beta = 1$. Aplicando la norma y usando su multiplicatividad,
    \[
    N(\alpha \beta) = N(\alpha)\,N(\beta) = N(1) = 1.
    \]
    Dado que $N(\alpha)$ y $N(\beta)$ son números enteros positivos (excepto si fueran cero, en cuyo caso no tendríamos una unidad), la única forma de que su producto sea 1 es que ambos valgan 1. Así, $N(\alpha)=1$.
\end{itemize}

En resumen, las unidades son exactamente aquellos elementos con norma igual a 1.

---

\noindent
\textbf{(c) Factorización de elementos no nulos ni unidades en irreducibles}

Para demostrar la factorización en irreducibles en $\mathbb{Z}[\sqrt{-n}]$, usamos el \textbf{principio de buena ordenación} en la norma.

\begin{itemize}
    \item Si $\alpha$ no es una unidad, entonces $N(\alpha) > 1$. Si $\alpha$ no es irreducible, se puede escribir como $\alpha = \beta \gamma$ con $\beta, \gamma$ no unidades.
    \item Como la norma es multiplicativa, tenemos que $N(\alpha) = N(\beta) N(\gamma)$, y por ser enteros positivos, se tiene $N(\beta), N(\gamma) < N(\alpha)$.
    \item Procedemos por inducción en la norma. Si todo elemento de norma menor que $N(\alpha)$ tiene factorización en irreducibles, entonces también lo tiene $\alpha$, pues sus factores $\beta$ y $\gamma$ se pueden descomponer en irreducibles.
    \item Aplicando el principio de buena ordenación, concluimos que todo elemento distinto de cero y no unidad en $\mathbb{Z}[\sqrt{-n}]$ se puede descomponer en irreducibles.
\end{itemize}

\[
\boxed{\text{Con esto, queda demostrada la factorización en irreducibles.}}
\]




\end{enumerate}

%%%%%%%%%%%%%%%%%%%%%%%%%%%%%%%%%%%%%%%%%%%%%%%%%%%%%%%%%%%%%%%%%%%%%%%%%%%%%%%%%%%%%%%%%%%%%%%%%%%%%%%%%%%%%%%%

\section*{Ejercicios de la clase}
\begin{enumerate}
    \item Sea $D$ un dominio entero y $F$ su campo de fracciones. Entonces, para cualquier polinomio $f(X) \in F[X]$, existe un polinomio $f_0(X) \in D[X]$ y un elemento $a \in D$ tal que:
    \[
    f(X) = \frac{f_0(X)}{a}.
    \]
    \begin{itemize}

        \item Dado que $D$ es un dominio entero, su campo de fracciones $F$ consiste en todas las fracciones de la forma $\frac{a}{b}$, donde $a, b \in D$ y $b \neq 0$. Consideremos el anillo de polinomios $F[X]$, cuyos elementos son expresiones de la forma:
        \[
        f(X) = \sum_{i=0}^{n} c_i X^i, \quad \text{con } c_i \in F.
        \]
        Queremos demostrar que cualquier polinomio en $F[X]$ puede escribirse como $f(X) = \frac{f_0(X)}{a}$, donde $f_0(X) \in D[X]$ y $a \in D$.

        \item Construcción de $f_0(X)$: Dado un polinomio $f(X) \in F[X]$, podemos escribir cada coeficiente $c_i$ en términos de elementos de $D$:
        \[
        c_i = \frac{a_i}{b_i}, \quad \text{con } a_i, b_i \in D, \quad b_i \neq 0.
        \]
        Sea $a$ el \textbf{mínimo común múltiplo} de los denominadores $b_0, b_1, \dots, b_n$, es decir,
        \[
        a = \operatorname{mcm}(b_0, b_1, \dots, b_n) \in D.
        \]
        Por la propiedad del mínimo común múltiplo, sabemos que $a$ es un múltiplo de cada $b_i$, lo que significa que existe $k_i \in D$ tal que:
        \[
        a = k_i b_i.
        \]
        Multiplicamos ambos lados por $a_i$, obteniendo:
        \[
        a a_i = k_i b_i a_i.
        \]
        Ahora, dividiendo por $b_i$ (que es distinto de cero en $D$):
        \[
        \frac{a a_i}{b_i} = k_i a_i.
        \]
        Dado que $k_i, a_i \in D$ y $D$ es un anillo, el producto $k_i a_i$ también pertenece a $D$. Definiendo $d_i = k_i a_i$, obtenemos:
        \[
        d_i = \frac{a a_i}{b_i} \in D.
        \]
        Definimos entonces el polinomio $f_0(X)$ en $D[X]$ como:
        \[
        f_0(X) = \sum_{i=0}^{n} d_i X^i.
        \]
        Por construcción, tenemos:
        \[
        f(X) = \sum_{i=0}^{n} c_i X^i = \sum_{i=0}^{n} \frac{a_i}{b_i} X^i = \sum_{i=0}^{n} \frac{d_i}{a} X^i = \frac{1}{a} \sum_{i=0}^{n} d_i X^i = \frac{f_0(X)}{a}.
        \]

        \item Conclusión: Hemos demostrado que cualquier polinomio en $F[X]$ puede escribirse como $f(X) = \frac{f_0(X)}{a}$ con $f_0(X) \in D[X]$ y $a \in D$. Esto implica que $D[X]$ es un subanillo de $F[X]$, ya que cada polinomio en $F[X]$ se obtiene como un polinomio en $D[X]$ dividido por un elemento de $D$.
        \[
        \square
        \]
    \end{itemize}

    \item Muestre que el polinomio \( p(x) = x^2 + x + 3 \) es irreducible en \( \mathbb{Q}[x] \).

\textbf{Solución:}

Para demostrar la irreducibilidad del polinomio \( p(x) = x^2 + x + 3 \) en \( \mathbb{Q}[x] \), utilizamos el siguiente resultado:

\textbf{Teorema 23.10 (Fraleigh, 7ma edición)}\\
\textit{
Sea \( f(x) \in F[x] \) y supongamos que \( f(x) \) tiene grado 2 o 3. Entonces \( f(x) \) es reducible sobre \( F \) si y solo si tiene una raíz en \( F \).
}

\textbf{Demostración:}\\
Supongamos que \( f(x) \) es reducible sobre \( F \). Entonces puede escribirse como el producto de dos polinomios no constantes en \( F[x] \), es decir,
\[
f(x) = g(x)\,h(x),
\]
donde \(\deg g(x) < \deg f(x)\) y \(\deg h(x) < \deg f(x)\). 
Dado que \( f(x) \) tiene grado 2 o 3, uno de los factores (por ejemplo, \( g(x) \)) debe ser de grado 1. Por lo tanto, 
\[
g(x) = x - a, \quad \text{para algún } a \in F.
\]
Como \( g(a) = 0 \), concluimos que \( a \) es una raíz de \( f(x) \). 
De esta manera, si \( f(x) \) es reducible sobre \( F[x] \), necesariamente tiene una raíz en \( F \).

Recíprocamente, si existe \( a \in F \) tal que \( f(a) = 0 \), entonces \( x - a \) es un factor de \( f(x) \), lo que muestra que \( f(x) \) es reducible.

\textbf{Aplicación al ejercicio:}\\
Para comprobar que \( p(x) = x^2 + x + 3 \) es irreducible en \(\mathbb{Q}[x]\), basta con verificar que no tiene raíces racionales. 
Resolviendo la ecuación cuadrática asociada,
\[
x = \frac{-1 \pm \sqrt{1 - 12}}{2} 
   = \frac{-1 \pm \sqrt{-11}}{2},
\]
se observa que \(\sqrt{-11}\notin \mathbb{Q}\). 
Por lo tanto, \( p(x) \) no posee raíces en \(\mathbb{Q}\) y, de acuerdo con el Teorema 23.10, es irreducible sobre \(\mathbb{Q}\).

    \item Determine los elementos de \( \mathbb{Q}[x] / \langle p(x) \rangle \).

Cada elemento de \( \mathbb{Q}[x] / \langle p(x) \rangle \) es una clase de equivalencia de la forma:
\[
a_0 + a_1 x + \cdots + a_{d-1} x^{d-1} + \langle p(x) \rangle
\]
donde \( p(x) \) es un polinomio de grado \( d \), y \( a_0, a_1, \dots, a_{d-1} \in \mathbb{Q} \).

En otras palabras, cada elemento puede representarse de manera única como un polinomio de grado menor que \( d \), es decir,
\[
\mathbb{Q}[x] / \langle p(x) \rangle \cong \mathbb{Q}[t],
\]
donde \( t \) es la clase residuo de \( x \) en el cociente, es decir, \( t = x + \langle p(x) \rangle \), y satisface la relación \( p(t) = 0 \).

\textbf{Ejemplo:}

Si \( p(x) = x^2 - 2 \), entonces los elementos de \( \mathbb{Q}[x] / \langle x^2 - 2 \rangle \) son:
\[
a + b t, \quad a, b \in \mathbb{Q},
\]
donde \( t^2 = 2 \). En este caso, el anillo cociente es isomorfo a \( \mathbb{Q}(\sqrt{2}) \).


    \item Encuentre el inverso multiplicativo para \( a + b t \) en \( \mathbb{Q}[x] / \langle p(x) \rangle \) con \( a + b t \neq 0 \).

\end{enumerate}
\end{document}

