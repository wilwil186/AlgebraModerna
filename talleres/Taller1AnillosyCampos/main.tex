\documentclass[12pt]{article}
\usepackage[utf8]{inputenc}
\usepackage{amsmath,amssymb,amsthm}
\usepackage{enumitem}
\usepackage{geometry}
\geometry{margin=2cm}
\usepackage{hyperref} % Opcional: para hipervínculos en el PDF

% Entornos theorem, proposition, etc. (por si los deseas)
\newtheorem{theorem}{Teorema}[section]
\newtheorem{proposition}[theorem]{Proposición}
\newtheorem{lemma}[theorem]{Lema}
\newtheorem{corollary}[theorem]{Corolario}
\theoremstyle{definition}
\newtheorem{definition}[theorem]{Definición}
\theoremstyle{remark}
\newtheorem{remark}[theorem]{Observación}
\newtheorem{example}[theorem]{Ejemplo}

% Comandos para atajos
\newcommand{\Z}{\mathbb{Z}}
\newcommand{\Q}{\mathbb{Q}}
\newcommand{\R}{\mathbb{R}}
\newcommand{\C}{\mathbb{C}}

\title{Ejercicios y Soluciones de Anillos y Campos}
\author{
    Camila Contreras (Código: 20182167055) \\
    Wilson Jerez (Código: 201181167034)
}
\date{
    Universidad Distrital Francisco José de Caldas \\
    Facultad de Ciencias Matemáticas y Naturales \\
    Programa Académico de Matemáticas
}

\begin{document}

\maketitle

\section*{Ejercicios y Soluciones}

\begin{enumerate}[label=\textbf{\arabic*.}]

%---------------------------------------------------------------------------------
\item \textbf{Sea $R$ un anillo y $a$ un elemento fijo de $R$. Sea $R_a$ el subanillo de $R$ que es la intersección de todos los subanillos de $R$ que contienen a $a$ (ver Ejercicio 49). El anillo $R_a$ es el subanillo de $R$ generado por $a$. Demuestra que el grupo abeliano $\langle R_a, + \rangle$ está generado (en el sentido de la Sección 7) por $\{a^n \mid n \in \mathbb{Z}^+\}$.}

\textbf{Solución:}\\
Por definición, cada subanillo de $R$ que contiene a $a$ debe contener también a todas las potencias $a^n$ (para $n \in \mathbb{Z}^+$), así como sus inversos aditivos. Por ende, el subanillo $R_a$ (intersección de todos esos subanillos) contiene $\{a^n \mid n \in \mathbb{Z}^+\}$. Sea $G$ el subgrupo aditivo generado por $S = \{a^n \mid n \in \mathbb{Z}^+\}$. Claramente, $G \subseteq \langle R_a, +\rangle$.

\medskip

Para mostrar que $G = R_a$, vemos que $G$ es cerrado bajo la multiplicación (gracias a la conmutatividad y la distributividad en $R$): el producto de dos sumas finitas de potencias de $a$ (incluyendo potencias negativas si uno considera los inversos aditivos) sigue siendo suma finita de potencias de $a$. De este modo, $G$ resulta ser un subanillo que contiene a $a$ y está contenido en $R_a$, así que $G = R_a$. 


%---------------------------------------------------------------------------------
\item \textbf{Resuelve la ecuación $3x = 2$ en el campo $\mathbb{Z}_7$ y en el campo $\mathbb{Z}_{23}$.}

\textbf{Solución:}\\
En $\mathbb{Z}_7$, necesitamos $x$ tal que $3x \equiv 2 \pmod{7}$. 
Se puede comprobar que $x = 3$ funciona: $3\cdot 3 = 9 \equiv 2 \pmod{7}$.  
Por lo tanto, la solución en $\mathbb{Z}_7$ es $x=3$.

\smallskip

En $\mathbb{Z}_{23}$, se desea $3x \equiv 2 \pmod{23}$. 
Podemos tantear o usar la inversa de $3$ en $\mathbb{Z}_{23}$: 
$3 \cdot 16 = 48 \equiv 2 \pmod{23}$.  
Así que $x = 16$ es la solución en $\mathbb{Z}_{23}$.


%---------------------------------------------------------------------------------
\item \textbf{Muestra que si $D$ es un dominio integral, entonces $\{n \cdot 1 \mid n \in \mathbb{Z}\}$ es un subdominio de $D$ contenido en cada subdominio de $D$.}

\textbf{Solución:}\\
Sea $R = \{n \cdot 1 \mid n \in \mathbb{Z}\}$. Observamos que si $n,m \in \mathbb{Z}$, 
\[
n\cdot 1 + m\cdot 1 = (n+m)\cdot 1 \quad\text{y}\quad 
(n\cdot 1)(m\cdot 1) = (nm)\cdot 1.
\]
Por lo tanto, $R$ está cerrado bajo la suma y el producto, y contiene el 1. Asimismo, $0 = 0\cdot 1$ está en $R$. Dado que $D$ no tiene divisores de 0 y $R$ hereda esa propiedad, $R$ tampoco tiene divisores de 0. 

En consecuencia, $R$ es un subdominio de $D$. Además, todo subdominio de $D$ que contenga $1$ debe contener a todos los enteros $n\cdot 1$, de manera que $R$ está contenido en cualquier otro subdominio.


%---------------------------------------------------------------------------------
\item \textbf{Dése la tabla de la multiplicación de grupo para los elementos de $\mathbb{Z}_{12}$ primos relativos con $12$. ¿A qué grupo de orden $4$ es isomorfo?}

\textbf{Comentario / Sugerencia:}  
Los elementos unidades en $\mathbb{Z}_{12}$ (es decir, los que son primos relativos con 12) son:
\[
U(\mathbb{Z}_{12}) \;=\; \{\,1, 5, 7, 11\}\,.
\]
Su orden es 4. Al construir la tabla de multiplicación (módulo 12), se observa que cada elemento es de orden 2 salvo la identidad, de modo que el grupo resultante es isomorfo a $\mathbb{Z}_2 \times \mathbb{Z}_2$ (el grupo de Klein).

\smallskip
\textbf{Tabla de multiplicación resumida (mod 12):}
\[
\begin{array}{c|cccc}
\cdot & 1 & 5 & 7 & 11 \\
\hline
1     & 1 & 5 & 7 & 11 \\
5     & 5 & 1 & 11 & 7 \\
7     & 7 & 11 & 1 & 5 \\
11    & 11 & 7 & 5 & 1 
\end{array}
\]
Se ve que cada elemento es su propio inverso (excepto la identidad $1$), lo que coincide con la estructura de Klein, $V_4 \cong \mathbb{Z}_2 \times \mathbb{Z}_2$.


%---------------------------------------------------------------------------------
\item \textbf{Describe el campo $F$ de cocientes del subdominio integral $D = \{n + mi \mid n, m \in \mathbb{Z}\}$ de $\mathbb{C}$.}

\textbf{Solución:}\\
El anillo $D$ consiste en todos los \emph{enteros gaussianos} $n + mi$, con $n,m \in \mathbb{Z}$. Su campo de cocientes (análogo a cómo $\mathbb{Q}$ se obtiene de $\mathbb{Z}$) es
\[
F \;=\; \Bigl\{\,q_1 + q_2\,i \;\Big|\; q_1,q_2 \in \mathbb{Q}\Bigr\}.
\]
Este campo se puede ver como tomar todos los elementos de $D$ y permitir divisiones por cualquier entero gaussiano no nulo. Al simplificar, se llega a números con partes real e imaginaria en $\mathbb{Q}$.


%---------------------------------------------------------------------------------
\item \textbf{Muéstrese, mediante un ejemplo, que un campo $F$ de cocientes de un subdominio propio $D'$ de un dominio entero $D$ también puede ser campo de cocientes de $D$.}

\textbf{Solución:}\\
Consideremos $D = \mathbb{Z}\bigl[\tfrac{1}{2}\bigr] = \left\{\frac{m}{2^n} : m \in \mathbb{Z}, n \in \mathbb{Z}_{\ge 0}\right\}$, que es un subanillo de $\mathbb{Q}$ (y por tanto un dominio entero). 
Sea $D' = \mathbb{Z}$, que claramente está contenido en $D$, pero $D' \subsetneq D$.

\smallskip

El campo de fracciones de $D'$ es $\mathbb{Q}$. Sin embargo, el campo de fracciones de $D$ también es $\mathbb{Q}$, porque al tomar cualquier cociente 
\[
  \frac{\frac{m_1}{2^{n_1}}}{\frac{m_2}{2^{n_2}}}
  \;=\;
  \frac{m_1}{2^{n_1}} \,\cdot\, \frac{2^{n_2}}{m_2}
  \;=\;
  \frac{m_1\,2^{n_2}}{m_2\,2^{n_1}}
  \;=\;
  \frac{m_1}{m_2}\cdot 2^{(n_2 - n_1)},
\]

eso es todavía un número racional. Por lo tanto, $K(D) = K(D') = \mathbb{Q}$, ilustrando que un subdominio propio puede compartir el mismo campo de fracciones que su superdominio.


%---------------------------------------------------------------------------------
\item \textbf{(Falso o Verdadero) sobre campos de cocientes de un dominio entero $D$:}

\begin{enumerate}[label=(\alph*)]
\item $\mathbb{Q}$ es un campo de cocientes de $\mathbb{Z}$.  
\textbf{Verdadero.} Ejemplo canónico.

\item $\mathbb{R}$ es un campo de cocientes de $\mathbb{Z}$.  
\textbf{Falso.} $\mathbb{Q}$ es el único (salvo isomorfismo) campo de fracciones de $\mathbb{Z}$, y $\mathbb{R}$ contiene números irracionales.

\item $\mathbb{R}$ es un campo de cocientes de $\mathbb{R}$.  
\textbf{Verdadero.} Si $D$ es un campo, su propio campo de cocientes es isomorfo a él mismo.

\item $\mathbb{C}$ es un campo de cocientes de $\mathbb{R}$.  
\textbf{Falso.} $\mathbb{R}$ ya es campo, su campo de fracciones se identifica con él mismo. No se “gana” nada pasando a $\mathbb{C}$.

\item Si $D$ es un campo, entonces cualquier campo de cocientes de $D$ es isomorfo a $D$.  
\textbf{Verdadero.}

\item El hecho de que $D$ no tenga divisores de $0$ se usó muchas veces en la construcción del campo de cocientes.  
\textbf{Verdadero.} Se requiere que $b\neq 0$ no se anule con ningún otro factor para que $\frac{a}{b}$ tenga sentido unívoco.

\item Todo elemento de un dominio entero $D$ es una unidad en un campo $F$ de cocientes de $D$.  
\textbf{Falso.} El $0$ no puede invertirse. Solo los no ceros de $D$ se vuelven unidades en $F$.

\item Todo elemento distinto de cero de un dominio entero $D$ es una unidad en un campo $F$ de cocientes de $D$.  
\textbf{Verdadero.}

\item Un campo de cocientes $F'$ de un subdominio $D'$ de un dominio entero $D$ puede considerarse subcampo de algún campo de cocientes de $D$.  
\textbf{Verdadero.} Existen monomorfismos naturales entre los campos de fracciones.

\item Todo campo de cocientes de $\mathbb{Z}$ es isomorfo a $\mathbb{Q}$.  
\textbf{Verdadero.}
\end{enumerate}


%---------------------------------------------------------------------------------
\item \textbf{Sea \(R\) un anillo conmutativo no nulo, y sea \(T\) un subconjunto no vacío de \(R\) cerrado bajo la multiplicación y que no contiene ni 0 ni divisores de 0. Partiendo de \(R \times T\) y siguiendo la construcción análoga a la de fracciones, se obtiene un anillo parcial de cocientes \(Q(R,T)\).}

\begin{enumerate}[label=(\alph*)]
\item Muestra que \(Q(R, T)\) tiene unidad aunque \(R\) no la tenga.  
\item En \(Q(R, T)\), cada elemento no nulo de \(T\) es una unidad.
\end{enumerate}

\textbf{Solución:}\\
\begin{enumerate}[label=(\alph*)]
\item Dado que $T$ es no vacío, elige $a \in T$. Entonces, el elemento $[(a,a)]$ en $Q(R,T)$ actúa como el 1: para todo $[(b,c)] \in Q(R,T)$,
\[
[(a,a)] \cdot [(b,c)] = [(ab, ac)] \;\sim\; [(b,c)],
\]
pues $abc = acb$ en un anillo conmutativo. Por lo tanto, $[(a,a)]$ es la unidad en $Q(R,T)$.

\item Si $a \in T$ y $a \neq 0$, entonces $[(a,a)]$ está en $Q(R,T)$. Para su inverso, se toma $[(a,aa)]$ o $[(aa,a)]$, según convenga. Se verifica que
\[
[(a,a)] \cdot [(aa,a)] 
= [(aaa, aa a)]
= [(a,a)],
\]
y esto muestra que cada $a\neq 0$ en $T$ se vuelve una unidad en $Q(R,T)$.
\end{enumerate}


%---------------------------------------------------------------------------------
\item \textbf{Encuentra todos los ideales $N$ de $\mathbb{Z}_{12}$. En cada caso, calcula $\mathbb{Z}_{12}/N$.}

\textbf{Solución:}\\
En $\mathbb{Z}_{12}$, los ideales (subgrupos aditivos estables por la multiplicación por elementos de $\mathbb{Z}_{12}$) son exactamente los generados por divisores de 12. Por notación cíclica:

\begin{align*}
\langle 0 \rangle &= \{0\}, \\
\langle 1 \rangle &= \{0,1,2,\dots,11\} = \mathbb{Z}_{12}, \\
\langle 2 \rangle &= \{0,2,4,6,8,10\}, \\
\langle 3 \rangle &= \{0,3,6,9\}, \\
\langle 4 \rangle &= \{0,4,8\}, \\
\langle 6 \rangle &= \{0,6\}.
\end{align*}


\smallskip
- $N = \langle 0 \rangle$:  $\mathbb{Z}_{12}/N \cong \mathbb{Z}_{12}$.
- $N = \langle 1 \rangle$:  $\mathbb{Z}_{12}/N \cong \{0\}$ (el anillo trivial).
- $N = \langle 2 \rangle$:  $\mathbb{Z}_{12}/N \cong \mathbb{Z}_2$.
- $N = \langle 3 \rangle$:  $\mathbb{Z}_{12}/N \cong \mathbb{Z}_3$.
- $N = \langle 4 \rangle$:  $\mathbb{Z}_{12}/N \cong \mathbb{Z}_4$.
- $N = \langle 6 \rangle$:  $\mathbb{Z}_{12}/N \cong \mathbb{Z}_2$.


%---------------------------------------------------------------------------------
\item \textbf{Determínense todos los ideales de $\mathbb{Z} \times \mathbb{Z}$.}

\textbf{Solución:}\\
Los ideales de \(\mathbb{Z} \times \mathbb{Z}\) son exactamente de la forma 
\[
n\mathbb{Z} \times m\mathbb{Z}, 
\quad \text{con } n, m \in \mathbb{Z}_{\geq 0}.
\]
\textit{Esbozo de demostración:} 
\begin{itemize}
\item Cada $n\Z \times m\Z$ es un ideal: está cerrado bajo suma y la multiplicación por cualquier elemento de $\Z \times \Z$. 
\item Si $I$ es un ideal de $\Z \times \Z$, sus proyecciones sobre cada coordenada, 
\[
I_1 = \{\,x \mid (x,y)\in I\},\quad 
I_2 = \{\,y \mid (x,y)\in I\},
\]
son ideales en $\Z$. Pero en $\Z$, los únicos ideales son de la forma $n\Z$. Así pues $I_1 = n\Z$ e $I_2 = m\Z$ para algunos $n,m\ge 0$. Se ve luego que $I\subseteq n\Z \times m\Z$ y, por la posibilidad de generar con $(n,0)$ y $(0,m)$, en realidad $I = n\Z \times m\Z$.
\end{itemize}


%---------------------------------------------------------------------------------
\item \textbf{Si $A$ y $B$ son ideales de un anillo $R$, se define } 
\[
A + B = \{\,a + b \mid a \in A,\, b \in B\}.
\]
\begin{enumerate}
\item Demuestra que $A + B$ es un ideal.
\item Demuestra que $A \subseteq A + B$ y $B \subseteq A + B$.
\end{enumerate}

\textbf{Solución:}\\
\begin{enumerate}[label=(\alph*)]
\item Sea $x = a_1 + b_1$ y $y = a_2 + b_2$, con $a_1,a_2\in A$ y $b_1,b_2\in B$. Entonces
\[
x + y = (a_1 + b_1) + (a_2 + b_2) = (a_1 + a_2) + (b_1 + b_2) \;\in\; A + B,
\]
así que está cerrado bajo suma. Si $r\in R$, 
\[
r\cdot(a_1 + b_1) = ra_1 + rb_1 \in A+B
\quad\text{y}\quad
(a_1 + b_1)\cdot r = a_1r + b_1r \in A+B
\]
pues $ra_1,a_1r\in A$ y $rb_1,b_1r \in B$, al ser $A,B$ ideales. Se verifica también que $0 \in A+B$ y los inversos aditivos están dentro. Por tanto, $A+B$ es un ideal.

\item Claramente $a = a+0 \in A+B$ para todo $a\in A$, y $b=0+b \in A+B$ para todo $b\in B$. Por ende, $A,B \subseteq A+B$.
\end{enumerate}


%---------------------------------------------------------------------------------
\item \textbf{Demuestra que el anillo de matrices $M_2(\mathbb{Z}_2)$ es un anillo simple (sin ideales propios no triviales).}

\textbf{Solución (Esbozo):}\\
Sea $R = M_2(\mathbb{Z}_2)$. Este anillo \textit{no} es conmutativo, pero igual consideramos sus ideales bilaterales. Observamos que si un ideal $I$ contiene al menos una de las matrices elementales (las que tienen un $1$ en una sola posición y $0$ en otras), entonces mediante multiplicaciones y sumas se generan todas las demás matrices elementales, y por ende, todo $R$. Así que cualquier ideal no trivial debe ser todo el anillo.

\smallskip

En concreto, las matrices elementales 
\[
E_{11} = \begin{pmatrix} 1 & 0 \\ 0 & 0 \end{pmatrix},\quad
E_{12} = \begin{pmatrix} 0 & 1 \\ 0 & 0 \end{pmatrix},\quad
E_{21} = \begin{pmatrix} 0 & 0 \\ 1 & 0 \end{pmatrix},\quad
E_{22} = \begin{pmatrix} 0 & 0 \\ 0 & 1 \end{pmatrix},
\]
al multiplicarlas de diversas formas (a izquierda o derecha), generan el resto de matrices. Cualquier ideal que contenga una de ellas termina conteniéndolas todas. Concluimos que $M_2(\mathbb{Z}_2)$ no tiene ideales bilaterales propios no triviales y, por tanto, es simple.


%---------------------------------------------------------------------------------
\item \textbf{Encuentra:}
\begin{enumerate}[label=\alph*)]
\item Un ideal maximal de $\mathbb{Z} \times \mathbb{Z}$. 
\item Un ideal primo de $\mathbb{Z} \times \mathbb{Z}$ que no sea maximal. 
\item Un ideal propio de $\mathbb{Z} \times \mathbb{Z}$ que no sea primo.
\end{enumerate}

\textbf{Solución:}\\
Recordemos que en $\mathbb{Z} \times \mathbb{Z}$ los ideales son $n\mathbb{Z}\times m\mathbb{Z}$. 

\begin{enumerate}[label=(\alph*)]
\item \textbf{Ideal maximal:}  
   Por ejemplo, $p\mathbb{Z}\times \mathbb{Z}$, donde $p$ es primo. El cociente
   \[
   (\mathbb{Z}\times\mathbb{Z})/(p\mathbb{Z}\times \mathbb{Z})
   \;\cong\;
   \mathbb{Z}/p\mathbb{Z}
   \times
   \mathbb{Z}/\mathbb{Z}
   \;\cong\; \mathbb{Z}/p\mathbb{Z},
   \]
   que es un cuerpo, de modo que es maximal. Concretamente, $2\mathbb{Z}\times \mathbb{Z}$ es un ejemplo.

\item \textbf{Ideal primo pero no maximal:}  
   Un ejemplo es $0\times \mathbb{Z}$. El cociente
   \[
   (\mathbb{Z}\times \mathbb{Z})/(0\times \mathbb{Z})
   \;\cong\;
   \mathbb{Z},
   \]
   y $\mathbb{Z}$ es un dominio entero (sin ser un cuerpo). Por lo tanto, $0\times \mathbb{Z}$ es primo sin ser maximal.

\item \textbf{Ideal propio no primo:}  
   Ejemplo: $2\mathbb{Z}\times 3\mathbb{Z}$. Su cociente es
   \[
   (\mathbb{Z}\times \mathbb{Z})/(2\mathbb{Z}\times 3\mathbb{Z})
   \;\cong\;
   \mathbb{Z}/2\mathbb{Z}\times \mathbb{Z}/3\mathbb{Z},
   \]
   un anillo con divisores de cero. Por ende no es un dominio, así que el ideal no es primo.
\end{enumerate}


%---------------------------------------------------------------------------------
\item \textbf{Sea \( R \) un anillo conmutativo con unidad de característica prima \( p \). Demuestra que el mapa \( \varphi_p : R \rightarrow R \) dado por \( \varphi_p(a) = a^p \) es un homomorfismo (el homomorfismo de Frobenius).}

\textbf{Solución:}\\
Por la expansión binomial, 
\[
(a+b)^p \;=\; \sum_{k=0}^{p} \binom{p}{k} a^k b^{p-k}.
\]
Cuando $p$ es primo, todos los coeficientes $\binom{p}{k}$ para $1 \le k \le p-1$ son múltiplos de $p$. En un anillo de característica $p$, esos coeficientes se anulan. Así,
\[
(a+b)^p \;=\; a^p + b^p.
\]
Además, siendo $R$ conmutativo, $(ab)^p = a^p b^p$. Esto prueba que
\[
\varphi_p(a+b) = (a+b)^p = a^p + b^p = \varphi_p(a) + \varphi_p(b),
\quad
\varphi_p(ab) = (ab)^p = a^p b^p = \varphi_p(a)\varphi_p(b).
\]
Por tanto, $\varphi_p$ es un homomorfismo de anillos (conocido como homomorfismo de Frobenius).


%---------------------------------------------------------------------------------
\item \textbf{Demuestra que $\phi : \mathbb{C} \rightarrow M_2(\mathbb{R})$ dada por 
\[
\phi(a + bi) = \begin{pmatrix} a & b \\ -b & a \end{pmatrix}
\]
es un isomorfismo sobre su imagen $\phi[\mathbb{C}]$.}

\textbf{Solución:}\\
Para $z_1 = a+bi$ y $z_2 = c+di$ en $\mathbb{C}$,
\[
\phi(z_1 + z_2) 
= \phi((a+c) + (b+d)i) 
= \begin{pmatrix} a+c & b+d \\ -(b+d) & a+c \end{pmatrix}
= \begin{pmatrix} a & b \\ -b & a \end{pmatrix}
 + \begin{pmatrix} c & d \\ -d & c \end{pmatrix}
= \phi(z_1) + \phi(z_2).
\]
Asimismo,
\[
\phi(z_1 z_2) 
= \phi((a+bi)(c+di)) 
= \phi((ac-bd) + (ad+bc)i)
= \begin{pmatrix} ac-bd & ad+bc \\ -(ad+bc) & ac-bd \end{pmatrix},
\]
y se comprueba que 
\[
\phi(z_1)\,\phi(z_2)
= \begin{pmatrix} a & b \\ -b & a \end{pmatrix}
  \begin{pmatrix} c & d \\ -d & c \end{pmatrix}
= \begin{pmatrix} ac-bd & ad+bc \\ -(ad+bc) & ac-bd \end{pmatrix}.
\]
Esto demuestra que $\phi$ es un homomorfismo de anillos. Resulta inyectivo (si $\phi(a+bi) = 0$, entonces $a=b=0$), y la imagen $\phi[\mathbb{C}]$ es un subanillo de $M_2(\mathbb{R})$. Por tanto, $\phi$ es un isomorfismo entre $\mathbb{C}$ y el subanillo \(\phi[\mathbb{C}]\subseteq M_2(\mathbb{R})\).


%---------------------------------------------------------------------------------
\item \textbf{(Falso o Verdadero) sobre DFU, DIP, etc.}

\begin{enumerate}[label=(\alph*)]
\item \textbf{Todo campo es un DFU (Dominio de Factorización Única).}  

\textbf{Verdadero.} En un campo, todo elemento no nulo es unidad y las “factorizaciones” se reducen a $a = a \cdot 1$.

\item \textbf{Todo campo es un DIP (Dominio de Ideales Principales).}  

\textbf{Verdadero.} En un campo $K$, los únicos ideales son $(0)$ y $K$, ambos principales.

\item \textbf{Todo DIP es un DFU.}  

\textbf{Verdadero.} Es un teorema estándar de álgebra conmutativa: los dominios de ideales principales (PID) son dominios de factorización única (UFD).

\item \textbf{Todo DFU es un DIP.}  

\textbf{Falso.} Ejemplo: $k[x,y]$ (polinomios en dos variables sobre un campo $k$) es UFD pero no PID.

\item \textbf{$\mathbb{Z}[x]$ es un DFU.}  

\textbf{Verdadero.} Si $R$ es UFD, entonces $R[x]$ también es UFD (caso particular: $R=\mathbb{Z}$).

\item \textbf{Cualesquiera dos irreducibles en cualquier DFU son asociados.}  

\textbf{Falso.} Ejemplo: en $\mathbb{Z}$, $2$ y $3$ son irreducibles pero no asociados.

\item \textbf{Si $D$ es un DIP, entonces $D[x]$ es un DIP.}  

\textbf{Falso.} Por ejemplo, $\mathbb{Z}$ es DIP, pero $\mathbb{Z}[x]$ no lo es.

\item \textbf{Si $D$ es un DFU, entonces $D[x]$ es un DFU.}  

\textbf{Verdadero.} Resultado que se demuestra usando el Lema de Gauss y propiedades de factorización.

\item \textbf{En cualquier DFU, si $p \mid a$ para un irreducible $p$, entonces $p$ aparece en toda factorización de $a$.}  

\textbf{Verdadero.} En un DFU, “irreducible” coincide con “prime”, y la divisibilidad de $p$ sobre $a$ implica que $p$ aparezca en la factorización única de $a$.

\item \textbf{Un DFU no tiene divisores de $0$.}

\textbf{Verdadero.} Todo DFU es, de hecho, un dominio, por lo que no admite divisores de cero.
\end{enumerate}

\end{enumerate}

\end{document}



