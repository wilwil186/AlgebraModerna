\documentclass[12pt]{article}
\usepackage[utf8]{inputenc}
\usepackage{amsmath, amssymb}

\title{Parcial II - Anillos y Campos}
\author{Universidad Distrital Francisco José de Caldas}
\date{}

\begin{document}

\maketitle

\begin{enumerate}
    \item ¿El polinomio \(x^2 + 3\) es irreducible en \(\mathbb{Z}_7\)?
    \begin{itemize}
        \item Para determinar si el polinomio $ x^2 + 3 $ es irreducible en $ \mathbb{Z}_7 $, podemos usar el siguiente criterio:
        
        \textbf{Teorema (Fraleigh, 7\textsuperscript{a} edición, Teorema 23.10):}  
        Un polinomio cuadrático o cúbico $ f(x) $ sobre un campo finito $ \mathbb{F} $ es reducible si y solo si tiene una raíz en $ \mathbb{F} $.
    
        \textbf{Paso 1: Evaluar el polinomio en los elementos de $ \mathbb{Z}_7 $}
    
        Calculamos $ f(a) = a^2 + 3 $ para $ a \in \mathbb{Z}_7 $:
    
        \[
        \begin{aligned}
        f(0) &= 0^2 + 3 = 3 \not\equiv 0 \pmod{7} \\
        f(1) &= 1^2 + 3 = 4 \not\equiv 0 \pmod{7} \\
        f(2) &= 2^2 + 3 = 4 + 3 = 7 \equiv 0 \pmod{7} \\
        f(3) &= 3^2 + 3 = 9 + 3 = 12 \equiv 5 \pmod{7} \\
        f(4) &= 4^2 + 3 = 16 + 3 = 19 \equiv 5 \pmod{7} \\
        f(5) &= 5^2 + 3 = 25 + 3 = 28 \equiv 0 \pmod{7} \\
        f(6) &= 6^2 + 3 = 36 + 3 = 39 \equiv 4 \pmod{7} \\
        \end{aligned}
        \]
    
        \textbf{Paso 2: Concluir sobre la reducibilidad}
    
        Observamos que $ f(2) \equiv 0 \pmod{7} $ y $ f(5) \equiv 0 \pmod{7} $, lo que significa que $ x^2 + 3 $ tiene raíces en $ \mathbb{Z}_7 $, específicamente $ x = 2 $ y $ x = 5 $. Según el teorema citado, esto implica que $ x^2 + 3 $ es \textbf{reducible} en $ \mathbb{Z}_7 $.
    
        Por lo tanto, el polinomio $ x^2 + 3 $ \textbf{no es irreducible en} $ \mathbb{Z}_7 $.
    \end{itemize}
    
    \item Dé un ejemplo de un Dominio de Factorización Única que no sea Dominio Euclidiano. 
    \textit{(Es evidente que debe mostrar el porqué no lo es).}
    Para factorizar el polinomio \( x^4 + 4 \) en \( \mathbb{Z}_5[x] \), seguimos los siguientes pasos:

\smallskip

\textbf{Paso 1: Expresión del polinomio en \( \mathbb{Z}_5[x] \)}

Dado que estamos en el cuerpo finito \( \mathbb{Z}_5 \), el número 4 es equivalente a \(-1\), por lo que podemos reescribir:

\[
x^4 + 4 \equiv x^4 - 1 \pmod{5}
\]

Observamos que esto se asemeja a una diferencia de cuadrados:

\[
x^4 - 1 = (x^2 - 1)(x^2 + 1).
\]

\smallskip

\textbf{Paso 2: Factorización de \( x^2 - 1 \) y \( x^2 + 1 \) en \( \mathbb{Z}_5[x] \)}

En \( \mathbb{Z}_5 \), las raíces de \( x^2 - 1 = 0 \) son los valores \( x = \pm 1 \). Esto nos da:

\[
x^2 - 1 = (x - 1)(x + 1).
\]

Ahora, examinemos \( x^2 + 1 \). En \( \mathbb{Z}_5 \), buscamos raíces de \( x^2 + 1 = 0 \), lo que equivale a encontrar soluciones para \( x^2 \equiv -1 \equiv 4 \pmod{5} \). Como \( 2^2 \equiv 4 \pmod{5} \), tenemos que \( x^2 + 1 \) tiene raíces en \( x = \pm 2 \), lo que nos da la factorización:

\[
x^2 + 1 = (x - 2)(x + 2).
\]

\smallskip

\textbf{Paso 3: Factorización completa en factores lineales}

Uniendo ambas factorizaciones, obtenemos:

\[
x^4 + 4 = (x - 1)(x + 1)(x - 2)(x + 2) \quad \text{en } \mathbb{Z}_5[x].
\]

Por lo tanto, la factorización completa en factores lineales en \( \mathbb{Z}_5[x] \) es:

\[
(x - 1)(x + 1)(x - 2)(x + 2).
\]

    
    \item Demuestre que en todo dominio euclidiano, para cualesquiera dos elementos \(a\) y \(b\), existe el máximo común divisor.
    
    \item Demuestre que todo Dominio Euclidiano es un Dominio de Factorización Única.
    
    \item Sea \(p\) un primo de la forma \(4n + 3\). Muestre que \(\mathbb{Z}_p[i]\) es un campo. 
    \textit{Sugerencia: Use el hecho de que los primos de esta forma no se pueden escribir como suma de cuadrados en los enteros.}
\end{enumerate}

\end{document}
