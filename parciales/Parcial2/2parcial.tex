\documentclass[12pt]{article}
\usepackage[utf8]{inputenc}
\usepackage{amsmath, amssymb}

\title{Parcial II - Anillos y Campos}
\author{Universidad Distrital Francisco José de Caldas}
\date{}

\begin{document}

\maketitle

\begin{enumerate}
    \item ¿El polinomio \(x^2 + 3\) es irreducible en \(\mathbb{Z}_7\)?
    \begin{itemize}
        \item Para determinar si el polinomio $ x^2 + 3 $ es irreducible en $ \mathbb{Z}_7 $, podemos usar el siguiente criterio:
        
        \textbf{Teorema (Fraleigh, 7\textsuperscript{a} edición, Teorema 23.10):}  
        Un polinomio cuadrático o cúbico $ f(x) $ sobre un campo finito $ \mathbb{F} $ es reducible si y solo si tiene una raíz en $ \mathbb{F} $.
    
        \textbf{Paso 1: Evaluar el polinomio en los elementos de $ \mathbb{Z}_7 $}
    
        Calculamos $ f(a) = a^2 + 3 $ para $ a \in \mathbb{Z}_7 $:
    
        \[
        \begin{aligned}
        f(0) &= 0^2 + 3 = 3 \not\equiv 0 \pmod{7} \\
        f(1) &= 1^2 + 3 = 4 \not\equiv 0 \pmod{7} \\
        f(2) &= 2^2 + 3 = 4 + 3 = 7 \equiv 0 \pmod{7} \\
        f(3) &= 3^2 + 3 = 9 + 3 = 12 \equiv 5 \pmod{7} \\
        f(4) &= 4^2 + 3 = 16 + 3 = 19 \equiv 5 \pmod{7} \\
        f(5) &= 5^2 + 3 = 25 + 3 = 28 \equiv 0 \pmod{7} \\
        f(6) &= 6^2 + 3 = 36 + 3 = 39 \equiv 4 \pmod{7} \\
        \end{aligned}
        \]
    
        \textbf{Paso 2: Concluir sobre la reducibilidad}
    
        Observamos que $ f(2) \equiv 0 \pmod{7} $ y $ f(5) \equiv 0 \pmod{7} $, lo que significa que $ x^2 + 3 $ tiene raíces en $ \mathbb{Z}_7 $, específicamente $ x = 2 $ y $ x = 5 $. Según el teorema citado, esto implica que $ x^2 + 3 $ es \textbf{reducible} en $ \mathbb{Z}_7 $.
    
        Por lo tanto, el polinomio $ x^2 + 3 $ \textbf{no es irreducible en} $ \mathbb{Z}_7 $.
    \end{itemize}
    
    \item Dé un ejemplo de un Dominio de Factorización Única que no sea Dominio Euclidiano. 
    \textit{(Es evidente que debe mostrar el porqué no lo es).}
    \textbf{Ejemplo:} Consideremos el anillo $\mathbb{Z}[x]$ de polinomios en una variable con coeficientes enteros.

\begin{enumerate}
    \item \textbf{$\mathbb{Z}[x]$ es un DFU:}
    
    Se sabe que $\mathbb{Z}$ es un dominio de factorizaci\'on unica (DFU) y, adem\'as, el anillo de polinomios sobre un DFU tambi\'en es DFU. Por lo tanto, $\mathbb{Z}[x]$ es un DFU.
    
    \item \textbf{$\mathbb{Z}[x]$ no es un Dominio Euclidiano:}
    
    Recordemos que todo dominio euclidiano es, en particular, un dominio principal de ideales (DIP). Demostraremos que $\mathbb{Z}[x]$ no es PID, por lo que no puede ser euclidiano.
    
    Consideremos el ideal
    \[
    I = \langle 2, x \rangle = \{2f(x) + xg(x) \mid f(x),g(x) \in \mathbb{Z}[x]\}.
    \]
    
    Supongamos, buscando una contradicci\'on, que $I$ es principal. Entonces existir\'ia un polinomio $h(x) \in \mathbb{Z}[x]$ tal que
    \[
    I = \langle h(x) \rangle.
    \]
    Esto implicar\'ia que $h(x)$ divide a ambos generadores $2$ y $x$.
    
    Analicemos las posibilidades para $h(x)$:
    
    \begin{itemize}
        \item \textbf{Si $h(x)$ es una unidad} (es decir, $h(x)=\pm 1$):  
        Entonces $\langle h(x) \rangle = \mathbb{Z}[x]$. Sin embargo, en este caso $1 \in \mathbb{Z}[x]$ estar\'ia en $I$, lo cual es falso, ya que no es posible expresar $1$ como una combinaci\'on lineal entera de $2$ y $x$.
    
        \item \textbf{Si $h(x)$ es asociado a $2$} (es decir, $h(x)=\pm 2$):  
        En este caso, aunque $h(x)$ divide a $2$, se debe verificar si tambi\'en divide a $x$. Pero no es posible que $2$ divida a $x$ en $\mathbb{Z}[x]$ ya que, de haberlo, existir\'ia un polinomio $q(x) \in \mathbb{Z}[x]$ tal que
        \[
        x = 2q(x).
        \]
        Esto implica que todos los coeficientes de $q(x)$ ser\'ian fraccionarios, lo cual es imposible en $\mathbb{Z}[x]$.
    
        \item \textbf{Si $\deg(h(x)) \ge 1$:}  
        Entonces $h(x)$ es un polinomio no constante. Pero en ese caso, $h(x)$ no puede dividir al entero $2$ (que es de grado $0$), a menos que $2$ sea producto de $h(x)$ por alg\'un polinomio, lo que no es posible.
    \end{itemize}
    
    Dado que ninguna de las opciones permite que $h(x)$ divida simult\'aneamente a $2$ y a $x$, concluimos que el ideal $I = \langle 2, x \rangle$ \textbf{no es principal}.
\end{enumerate}

\textbf{Conclusi\'on:} Aunque $\mathbb{Z}[x]$ es un dominio de factorizaci\'on \unica, no es un dominio euclidiano, ya que no es un dominio principal (el ideal $\langle 2, x \rangle$ no es principal). Este ejemplo ilustra que la propiedad de ser DFU no implica ser DE.

    
    \item Demuestre que en todo dominio euclidiano, para cualesquiera dos elementos \(a\) y \(b\), existe el máximo común divisor.
    
    \item Demuestre que todo Dominio Euclidiano es un Dominio de Factorización Única.
    
    \item Sea \(p\) un primo de la forma \(4n + 3\). Muestre que \(\mathbb{Z}_p[i]\) es un campo. 
    \textit{Sugerencia: Use el hecho de que los primos de esta forma no se pueden escribir como suma de cuadrados en los enteros.}
\end{enumerate}

\end{document}
